\section{Styrmodul}

Styrmodulen tar emot kommandon från kartmodulen och kontrollerar, utifrån
dessa, robotplattformens motorer. Syftet är att avlasta kartmodulen från behöva
utföra PWM. Styrmodulen fungerar också som ett abstraktionslager mellan
kartmodul och plattform. Om man i framtiden vill byta plattform till en som
styrs på annat vis, behöver man bara byta ut styrmodulen.

\subsection{Hårdvara}

Styrmodulen består av en AVR ATmega8 som sitter kopplad mellan kartmodulen och
robotplattformens H-brygga. AVR ATmega8 har inbyggd PWM och UART. UART:en är
kopplad till kartmodulens USB-serieport och två PWM-utgångar är kopplade till
vardera PWM-ingång på H-bryggan. (En kanal styr motorerna på vänster sida och
en annan kanal styr motorerna på höger sida.) Två I/O-utgångar på AVR:en är
kopplade till vardera riktnings-ingång på H-bryggan.

En extern klocka som är gemensam för de båda AVR:erna i roboten.

\subsection{Mjukvara}

Styrmodulens mjukvara är väldigt enkel. Arbetet utförs i en avbrottsrutin som
anropas när det kommer en byte på serieporten. När styrmodulen tagit mot två
bytes kommer den att tolka dessa som signade heltal och justera de två
PWM-utgångarna (som anger hastighet) och de två I/O-utgångarna (som anger
riktning) därefter. En negativ byte försätter motorn i backriktning och en
positiv i framriktning. Bytens absolutbelopp anger hastigheten. MSB används
alltså för att styra motsvarande I/O-utgång och resten av bitarna i byte:n
används för att kontrollera pulsbredden på motsvarande PWM-utgång. 127 är full
fart frammåt, -128 är full back och 0 är stillastående.

\subsection{Kopplingsschema}
Se appendix \ref{styr-sch}.
