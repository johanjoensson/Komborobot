\section{Testning}
Integrationstestning görs för att kontrollera att alla gränssnitt fungerar som
de ska och att delsystemen svarar som förväntat på testdata. Vi kommer att
mestadels använda black-box testning men vid behov kommer vi att använda 
grey-box för att testa eventuella tillstånd som är svåra att nå.

Alla projektmedlemmar förväntas att göra sin egen enhetstestning alternativt se 
till att den blir gjord.

\subsection{Sensor- och kartenhet}
Sensorenheten är ganska simpel att testa då vi bara kan skicka in ett värde i
sensoränden och se om vi får ut ett bra värde när vi frågar efter det.

Kartenheten kan kontrolleras genom att skicka in en serie av sensorvärden och
se hur den omvandlar det till väggar på en karta.

\subsection{Styr- och kartenhet}
Styrenheten testas genom att alla olika styrkommandon skickas till den och vi
studerar dess utsignaler.

Bussen är enkelriktad så kartenheten kan inte skicka några meddelanden till
styrenheten.

\subsection{Robotsystemtest}
Vi systemtestar roboten genom att koppla ihop hela och se om den beter sig
specifikationsenligt när vi ger den kommandon, systemtestet är så beroende av
ett fungerande användargränssnitt i PC-mjukvaran att det inte kan genomföras
innan det är i alla fall så fullständigt att vi har ett kommandoradsgränssnitt.

\subsection{Systemintegrationstest}
Testning av hela systemet inklusive labyrint och PC-mjukvara. Vi kör runt
roboten i labyrinten och ser att kartan ritas korrekt på PC:n.
