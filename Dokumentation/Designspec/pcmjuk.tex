\section{PC mjukvara}
PC-mjukvaran har som uppgift att låta användaren kommunicera med roboten via ett
enkelt gränssnitt. Via PC-mjukvaran kan användaren få intressant information
från robotens olika moduler, t.ex avstånd till väggar eller vilket styrkommando
som nu utförs. PC-mjukvaran kommunicerar med roboten via blåtand.

\subsection{Implementation}

Mjukvaran är skriven i programspråket C och använder utöver Cs standardbibliotek
även gränssnittet BlueZ för att kommunicera via blåtand.

\subsection{Användande}

Användargränssnittet utgörs av ett program som i fjärrstyrt läge låter 
användaren skriva in önskat styrkommando åt roboten, exempelvis ''fram''. Detta
kommando kommer sedan utföras av roboten tills dess att användaren skriver in
ett nytt kommando åt roboten. Gränssnittet kommer även att visa vilket
styrkommando roboten för tillfället utför samt information från robotens
sensorer.

Då roboten befinner sig i autonomt läge kommer gränssnittet att kontinuerligt
uppdateras med de styrkommandon som roboten för tillfället utför samt
information från robotens sensorer. 

Gränssnittet är enkelt att använda, kommandona logiska och simpla och
informationen från roboten visas på ett tydligt och lättförståeligt sätt.

% \subsection{Implementation}
% 
% Mjukvaran är skriven i programspråken C samt Tcl och använder utöver Cs standardbibliotek
% även gränssnittet BlueZ för att kommunicera via blåtand.
% 
% \subsection{Användande}
% 
% Mjukvaran kommer utgöras av ett fönster där användaren kan se information
% skickad från robotens sensorer och dess styrenhet samt skicka styrkommandon till
% roboten.
% 
% I fjärrstyrt läge kan användaren välja att använda de pilknappar som finns i
% fönstret för att styra roboten eller så kan användaren använda piltangenterna på
% tangentbordet.
% 
% I autonomt läge kommer gränssnittet visa information från robotens sensorer samt
% vilket styrkommando som roboten just nu utför.
% 
% Gränssnittet är enkelt att använda, kommandona är logiska och simpla och
% informationen från roboten visas på ett tydligt och lättförståeligt sätt.
% 
