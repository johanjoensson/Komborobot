\section{Styrenhet}
Styrenhetens uppgift är att styra motorerna utifrån data som skickas från kommunikationsenheten.
Styrenheten ska kunna vara i två lägen, autonomt läge då den reglerar motorerna utifrån sensorvärden eller 
fjärrstyrt läge då den tar emot och behandlar styrkommandon.
\subsection{Hårdvara}
Styrmoodulen består av en AVR ATmega16 som ska placeras på robotplattformen.
Mikroprocessorns PWM-utgångar ska kopplas till PWM-ingångarna på robotplattformen för att styra motorerna.
För att styra riktning på motorerna kopplas två I/O-utgångar från mikroprocessorn till plattformens riktningsingångar.
\subsection{Mjukvara}
Mjukvaran kommer att fungera väldigt olika beroende på om roboten är i fjärrstyrt eller autonomt läge.
\subsubsection{Fjärrstyrt läge}
I fjärrstyrt läge så genereras ett avbrott när ett kommando tagits emot.
Varje kommando är en byte långt där de fyra mest signifikanta bitarna bestämmer vilket kommando som ska utföras
och de 4 minst signifikanta bitarna bestämmer hur kommandot ska utföras (t ex hur snabbt roboten ska svänga).
Uppdelningen kan ses i tabell \ref{styrbitar}.

\begin{table}[h] 
        \label{styrbitar}
        \begin{center}
                \begin{tabular}{| c | c |}
                        \hline
                        Kommando [4 bitar] & Hastighet [4 bitar] \\ \hline
                \end{tabular}
        \end{center}
        \caption{Bituppdelning av styrkommandon}
\end{table}
I avbrottsrutinen så tolkas kommandot och PWM-utgångarna och I/O-utgångarna som styr riktningarna på motorerna ställs.
Så om inget nytt avbrott kommer så fortsätter det senaste kommandot.
Kodningen av kommandon ses i tabell \ref{kodning}.
\begin{table}[h] 
        \label{kodning}
        \begin{tabular}{l l}
                \textbf{4 MSB} & \textbf{Kommando} \\
                0000 & Stopp \\
                0001 & Framåt \\
                0010 & Fram vänster \\
                0011 & Fram höger \\
                0100 & Rotera vänster \\
                0101 & Rotera höger \\
                0110 & Back \\
        \end{tabular}
        \caption{Kommandokodning}
\end{table}


\subsubsection{Autonomt läge}
\subsection{Reglering}
\subsection{Blockschema}
Se appendix %\ref{app:styrenhet}
