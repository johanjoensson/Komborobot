\section{Testning}
Tester av systemet ska göras både i större och mindre utförande för att undvika att tekniska problem och designmissar följer med under längre tid i projektet.

\subsection{Kontinuerliga tester}
Kontinuerliga tester ska ske regelbundet genom hela projektet. Dessa tester ska ske avslutningsvis efter varje aktivitet. Detta innebär att när man har avslutat en aktivitet i tidsplanen så ska man testa att det fungerar och uppfyller de kraven som är satta. först när man har gjort detta så kan man meddela till resten av gruppen att aktiviteten är slutförd. Om man gör tester och det visar sig att man inte når upp till kraven för aktiviteten men att man ändå vill fortsätta och bygga vidare på samma lösning så ska testresultaten rapporteras så att nästa som ska använda, eller utveckla, vet vilka fel som uppstod sist.

\subsection{Stora tester}
Tester i större skala ska göras i samband med att en modul är färdig. Testet ska då vara omfattande och alla kraven som påverkar modulen ska gås igenom grundligt. Speciellt viktigt är att testa de signaler och den kommunikation som sker med andra enheter. Skulle något av detta vara fel så måste det lösas samt rapporteras grundligt.

\subsection{Kalibrering}
När man gör tester för att kalibrerar systemet så ska man skriva upp de värden man har testat med, vilket utfall det fick och lämna några kortare kommentarer om varför det är bra dåligt. detta för att undvika att samma värden testas igen och för att man lättare ska kunna se samband och hitta optimala värden.