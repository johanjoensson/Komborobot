\section{Sensorenhet}

\subsection{Hårdvara}
Sensorenheten består av en AVR ATmega16 som kopplats till 3 avståndssensorer, de sensorer som hittar linjer på marken samt en display som visar avståndet till väggar. 

\subsection{Linjeföljarsensor}
Linjeföljarsensorn utgörs av en reflexsensormodul kopplad till en mux och en demux av typen 4067. Reflexsensormodulen består av 11 lysdioder och 11 ljuskänsliga transistorer.  Muxarna styrs av AVR:en och får lysdioderna och transistorerna att arbeta i par. Den ena muxen är kopplad till logiskt ett på sin ingång för att tända en lysdiod i taget, demuxen kopplar motsvarande transistor till A/D omvandlaren. Efter att en omvandling är klar kopplas muxarna om till nästa par och en ny omvandling kan starta.

analog multiplexer 4067B (16 kanaler)

filtrering? för varje transistor eller en för alla?
tid? behövs det set-up tid för att transistorn ska bli ledande?
4 pinnar
1 AD pinne

\subsubsection{Avståndssensorer}
Sensorenheten har 3 optiskaavståndsmätare av typen GP2Y0A02YK (20-150 cm). En placerad på vardera sida av roboten och en riktad frammåt. Dessa ger en analog utsignal som kopplas till AVR:ens AD omvandlare och de tar upp en pinne var.

filtrering?
Tid: ca 50 ms mellan mätningar
3 AD pinnar

\subsubsection{Display}
En alfanumerisk LED-display av typen HDSP-2112 används för att visa avstånd till väggarna när roboten är i labyrintläge. Den kopplas till Port ? på AVR:en, 8 pinnar.

vilka input har den?
8 pinnar

\subsection{Mjukvara}


\subsubsection{AD}

\subsubsection{Linjeföljare}

\subsubsubsection{Mux}

\subsubsection{Kommunikation}