% !TEX encoding = UTF-8 Unicode

%% --------------------------------------------------------------------------------------------------------------------------------------------
% ÖVERSIKT - Beskrivning av robotens funktioner och gränssnitt mellan robot och användare
%
%
% --mj 20120506
%% --------------------------------------------------------------------------------------------------------------------------------------------


\section{Översikt}

Man kan interagera med Komborobot på två sätt. Dels genom att använda knapparna och reglagen på robotens ovansida, dels genom datorprogrammets gränssnitt. 

\subsection{Robotens utseende}

Figur \ref{fig:robot1} visar roboten, sett från ovan. Ovanpå roboten finns en skjutspak (1), två knappar (2 och 3),  samt en display (4).   Förutom detta så sitter även robotens bluetooth-enhet (5) på samma platta. 

\nyBild{robot2.jpg}{Robotens gränssnitt mot användaren}{robot1}{0.5} %Bild på robot ovanifrån

\begin{enumerate}
\item{\bf Lägesspaken} bestämmer vilket styrläge roboten befinner sig i, d.v.s. om den befinner sig i \emph{autonomt} eller \emph{fjärrstyrt} läge. Spakens riktning för de två lägena kan ses i figur \ref{fig:spak}.
\item {\bf Startknappen} används endast i autonomt läge. Som namnet antyder används den för att starta roboten från stillastående. 
\item{\bf Reset-knappen} används för att få roboten att återgå till ursprungsläget, vilket i både autonomt och fjärrstyrt läge innebär att roboten stannar. Efter reset så startas roboten åter med antingen (2) eller kommando från PC, beroende på styrläge.
\item{\bf Displayen} visar avståndet till väggarna på robotens högra, vänstra samt främre sida. Värdena anges i cm. 
\item{\bf Bluetooth-enheten} används för att kommunicera med PC. Då roboten är ansluten till en PC så lyser BT-enhetens lysdiod grönt, i annat fall blinkar den rött. 
\end{enumerate}

\nyBild{spakar2.jpg}{Lägesspakens två lägen. (1) innebär autonomt och (2) fjärrstyrt läge.}{spak}{1.3} %Bild på robot ovanifrån

\subsection{Datorprogrammets gränssnitt}

Användatgränssnittet består av 2 olika program. input\_control som hanterar
hanterar användarens tangentbordstryckningar samt send\_receive som skickar
användarens intruktioner (genererade med input\_control) till Komborobot.

\nyBild{granssnitt.png}{Användargränssnittet send\_receive}{sendrec}{0.5}

send\_receive kommer på skärmen visa; ifall Komborobot kör i en labyrint, längs en linje eller ifall
den fjärrstyrs, information från de avståndssensorer som är monterade på
robotens sidor (4 st), vilken hastighet man skickade till roboten i samband med
förra kommandot, vilken trimnivå motorerna har, vilket specialkommando roboten
utför (endast i autonomt läge) samt ifall något felaktigt kommando har skickats
till styrenheten.

\begin{enumerate}
\item Avståndssensor, vänster fram på Komborobot.
\item Avståndssensor, vänster bak på Komborobot.
\item Avståndssensor, höger fram på Komborobot.
\item Avståndssensor, höger bak på Komborobot.
\item Labyrintläge, linjeföljarläge eller fjärrstyrt läge.
\item Senast skickad hastighet.
\item Trimnivå vänster motor.
\item Trimnivå höger motor.
\item Felaktigt kommando skickat till styrenheten.
\item Senast utförda specialkommandon.

\nyBild{input_control.png}{Användargränssnittet input\_control}{input}{1}
input\_control visar en fyrkant på skärmen, färgen på denna fykant visar det
nuvarande värdet på linjesensorkalibreringskonstanten. Denna fyrkant
\emph{måste} vara markerad för att man ska kunna styra roboten via PCn.

\item input\_control.

\end{enumerate}
