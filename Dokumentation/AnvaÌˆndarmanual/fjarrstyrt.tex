% !TEX encoding = UTF-8 Unicode

%% --------------------------------------------------------------------------------------------------------------------------------------------
% FJÄRRSTYRT LÄGE - Användning vid styrning via PC 
%
%
% --mj 20120506
%% --------------------------------------------------------------------------------------------------------------------------------------------


\section{Fjärrstyrt läge}
\subsection{PC}
I fjärrstyrt läge kan användaren styra Komborobot med hjälp av en PC.
Information från Komborobots avståndssensorer, Komborobots hastighet samt
motorernas trimnivåer kommer kontinuerligt att visas på datorskärmen.
\subsubsection{input\_control}
input\_control hanterar användarens knapptryckningar och genererar instruktioner
som skickas till Komborobot.

Nedan följer vilka knapptryckningar som genererar de olika kommandona:

w - Kör framåt med given hastighet.
s - Kör bakåt med given hastighet.
a - Rotera vänster med given hastighet.
d - Rotera höger med given hastighet.
q - Sväng vänster (mjuk kurva) med given hastighet.
e - Sväng höger (mjuk kurva) med given hastighet.
[Mellanslag] - Stopp.

[Pil upp] - Öka hastigheten till nästa kommando.
[Pil ner] - Minska hastigheten till nästa kommando.

[Pil höger] - Öka trimnivån på höger motor.
[Pil vänster] - Öka trimnivån på vänster motor.
o - Nollställ trimnivåerna.

u - Öka värdet på linjesensorkalibreringskonstanten.
i - Minska värdet på linjesensorkalibreringskonstanten.
c - Kalibrera linjesensorerna med linjesensorkalibreringskonstanten.

Då värdet på linjesensorkalibreringskonstanten ändras kommer bakgrunden på
input\_control att ändras för att approximativt spegla vid vilken färg vi går
från svart till vitt.

För att avsluta input\_control trycker man [escape], eller använder krysset på
fönstret.

\subsubsection{send\_receive}
I fjärrstyrt läge kommer send\_receive att visa hastighet på senast skickat
kommando, trimnivåerna på motorerna samt information från avståndssensorerna.

send\_receive avslutas genom att trycka [ctrl + c].
