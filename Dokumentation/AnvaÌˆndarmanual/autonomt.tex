% !TEX encoding = UTF-8 Unicode

%% --------------------------------------------------------------------------------------------------------------------------------------------
% AUTONOMT LÄGE - Beskrivning av användning vid autonom styrning 
%
%
% --mj 20120506
%% --------------------------------------------------------------------------------------------------------------------------------------------

\section{Autonomt läge}
I det autonoma läget så styr roboten sig själv. Roboten är designad för att kunna ta sig igenom en bana bestående av labyrinter samt linjer på marken. Banans utseende beskrivs i detalj i sektion \ref{sec:banan}.

\subsection{Start och stopp}
För att starta roboten i det autonoma läget så görs följande: (se figur \ref{fig:robot1} för nummerreferenser)

\begin{enumerate}
\item Slå på strömmen. (6)
\item Se till att roboten är ställd i autonomt läge. (1)
\item Tryck på reset för att sätta roboten i inaktivt läge. (3)
\item Roboten kan börja antingen på en linje eller i labyrinten. I linjeläge måste robotens linjesensorer stå över linjen, och roboten vara riktad i riktning med linjen. 
\item Körning startas med startknappen. (2)
\end{enumerate}

Roboten kommer att följa banan fram till dess att stoppkommando hittas. Man kan även stanna roboten manuellt med reset-kappen (3) eller genom att slå av strömmen (6). 

\subsection{Kalibrering}
Todo

\subsection{Displayen}
Displayen visar avståndet till väggarna på sidorna till höger, vänster och framför, givet i cm. Figur \ref{fig:display} visar displayens utseende. 

\nyBild{display2.jpg}{Displayens utseende.}{display}{1.0}

Displayen har två rader med tecken. Den övre raden visar framsensorns och de främre sidosensorernas värden, och den undre raden visar bakre sidosensorernas dito. Ett 'H' bredvid värdet indikerar högersensor, 'V' vänstersensor och 'F' sensorn riktad framåt. Avstånden visas med tre siffror, och ligger i intervallet 20 cm - 120 cm. Detta innebär att avstånd som ligger utanför detta intervall kommer att ge ett felaktigt värde på displayen. 

\subsection{PC}
I autonomt läge så kommer Komborobot att kontinuerligt skicka avstånd till
väggarna längs sidorna till PC. Då roboten utför ett specialkommando, då den
exempelvis kommer fram till en korsning,  kommer Komborobot skicka information om
detta till PC. Skulle något gå fel och styrenheten får in ett kommando den inte
känner igen kommer Komborobot att skicka ett meddelande till PC. Var gång
Komborobot skickar något till PC så inkluderas information om den följer en
linje eller om den kör i en labyrint.
\subsection{Banan}
\label{sec:banan}

Banan består av två moment, en labyrint att navigera i och en linje på marken att följa. Banan kan utformas så att den växlar mellan labyrint- och linjemoment ett godtyckligt antal gånger. Linjen tar vid mitt i utgången av en labyrint, och leder upp hela vägen till mitten av nästa ingång. Ett exempel på utformningen av en bana kan ses i fig. \ref{fig:bana1}.

\nyBild{bana2.jpg}{Ett exempel på utformningen av en bana.}{bana1}{0.5}

Roboten är designad för att navigera efter en svart linje på grått underlag. Linjens bredd ska vara i intervallet 14-18 mm. Linjen kan korsa sig själv i 90 graders korsningar, men kan ej dela sig. Svängradien ska vara minst 25 cm. 

För labyrintmomentet är roboten designad för en 80 cm bred bana, d.v.s. i raksträckorna i labyrinten så är avståndet mellan väggarna 80 cm. Roboten kan hantera svängar i form av vanliga 90\degree-kurvor, tre- samt fyrvägskorsningar. I de två senare fallen då fler vägvalsmöjligheter finns så kräver roboten markeringar i form av linjer på golvet. De olika markeringarna visas i figur \ref{fig:markeringar}

\nyBild{bild.jpg}{De olika riktningsmarkeringarnas utseende.}{markeringar}{1.0}

Banan slutar i ett linjemoment, och markeras med stoppsignalen ovan. 

