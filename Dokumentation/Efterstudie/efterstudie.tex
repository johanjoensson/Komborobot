% !TEX encoding = UTF-8 Unicode

\documentclass[a4paper,12pt]{article}
\usepackage[swedish]{babel}
\usepackage[utf8]{inputenc}
\usepackage{graphicx}
\usepackage{epstopdf}
\usepackage{gensymb}
%% Definitioner för LIPS-dokument

\usepackage[swedish]{babel}
\usepackage[utf8]{inputenc}
\usepackage[T1]{fontenc}
\usepackage{times}
\usepackage{ifthen}

\usepackage[margin=25mm]{geometry}

\usepackage{fancyhdr}
\pagestyle{fancy}
\lhead{}
\chead{\textbf{\LIPSprojekttitel}}
\rhead{\textbf{\textsl{LiTH}}\\\textbf{\LIPSdatum}}
\lfoot{\textbf{\LIPSkursnamn}\\\textbf{\LIPSdokumentansvarig}}
\cfoot{\textbf{\LIPSprojektgrupp}\\\textbf{\LIPSgruppepost}}
\rfoot{\textbf{\textsc{Lip}s}\\\textbf{Sida~\thepage}}

\setlength{\parindent}{0pt}
\setlength{\parskip}{1ex plus 0.5ex minus 0.2ex}


\newcommand{\twodigit}[1]{\ifthenelse{#1<10}{0}{}{#1}}
\newcommand{\dagensdatum}{\number\year-\twodigit{\number\month}-\twodigit{\number\day}}

%% ------------------------------------------
% NYBILD
% Skapar centrerad bild med caption
%   
% #1: Filens url relativt '/bilder/'
% #2:  Caption
% #3: Label
% #4: Skalning
%% ------------------------------------------
\newcommand{\nyBild}[4] 
{\begin{figure}[H]
  \centering
 \includegraphics[angle=0,scale=#4]{bilder/#1}
  \caption{#2}
  \label{fig:#3}
\end{figure}}



%%  Redefinitions of commands containing @
\makeatletter
\makeatother

\newcommand{\LIPStitelsida}{%
{\ }\vspace{45mm}
\begin{center}
  \textbf{\Huge \LIPSdokumenttyp}
\end{center}
\begin{center}
  {\Large Redaktör: \LIPSredaktor}
\end{center}
\begin{center}
  {\Large \textbf{Version \LIPSversion}}
\end{center}
\vfill
\begin{center}
  {\large Status}\\[1.5ex]
  \begin{tabular}{|*{3}{p{40mm}|}}
    \hline
    Granskad & \LIPSgranskare & \LIPSgranskatdatum \\
    \hline
    Godkänd & \LIPSgodkannare & \LIPSgodkantdatum \\
    \hline
  \end{tabular}
\end{center}
\newpage
}


\newenvironment{LIPSprojektidentitet}{%
{\ }\vspace{45mm}
\begin{center}
  {\Large PROJEKTIDENTITET}\\[0.5ex]
  {\small
  \LIPSartaltermin, \LIPSprojektgrupp\\
  Linköpings Tekniska Högskola, ISY
  }
\end{center}
\begin{center}
  {\small Gruppdeltagare}\\
%  \begin{tabular}{|p{30mm}|p{40mm}|p{35mm}|p{45mm}|}
  \begin{tabular}{|l|p{45mm}|p{25mm}|l|}
    \hline
    \textbf{Namn} & \textbf{Ansvar} & \textbf{Telefon} & \textbf{E-post} \\
    \hline
}%
{%
    \hline
  \end{tabular}
\end{center}
\begin{center}
  {\small
    \textbf{E-postlista för hela gruppen}: \LIPSgruppepost\\
    \textbf{Hemsida}: \LIPSgrupphemsida\\[1ex]
    \textbf{Kund}: \LIPSkund\\
    \textbf{Kontaktperson hos kund}: \LIPSkundkontakt\\
    \textbf{Kursansvarig}: \LIPSkursansvarig\\
    \textbf{Handledare}: \LIPShandledare\\
  }
\end{center}
\newpage
}
\newcommand{\LIPSgruppmedlem}[4]{\hline {#1} & {#2} & {#3} & {#4} \\}



\newenvironment{LIPSdokumenthistorik}{%
\begin{center}
  Dokumenthistorik\\[1ex]
  \begin{small}
    \begin{tabular}{|l|l|p{60mm}|l|l|}
      \hline
      \textbf{Version} & \textbf{Datum} & \textbf{Utförda förändringar} & \textbf{Utförda av} & \textbf{Granskad} \\
      }%
    {%
      \hline
    \end{tabular}
  \end{small}
\end{center}
}
\newcommand{\LIPSversionsinfo}[5]{\hline {#1} & {#2} & {#3} & {#4} & {#5} \\}

\newcounter{LIPSkravnummer}
\newcounter{LIPSunderkravnummer}[LIPSkravnummer]
\newenvironment{LIPSkravlista}{%
  \begin{tabular}{|p{25mm}|p{25mm}|p{85mm}|p{5mm}|}
    }%
  {%
    \hline
  \end{tabular}
}
\newcommand{\LIPSkrav}[3]{\hline\stepcounter{LIPSkravnummer}\textbf{Krav nr \arabic{LIPSkravnummer}} & \textbf{{#1}} & {#2} & \textbf{{#3}} \\}
\newcommand{\LIPSunderkrav}[3]{\hline\stepcounter{LIPSunderkravnummer}\textbf{Krav nr \arabic{LIPSkravnummer}\Alph{LIPSunderkravnummer}} & \textbf{{#1}} & {#2} & \textbf{{#3}} \\}





%%% Local Variables: 
%%% mode: latex
%%% TeX-master: "kravspec_mall"
%%% End: 



\newcommand{\LIPSartaltermin}{2012/VT}
\newcommand{\LIPSkursnamn}{TSEA27}

\newcommand{\LIPSprojekttitel}{Komborobot}

\newcommand{\LIPSprojektgrupp}{Grupp 17}
\newcommand{\LIPSgruppepost}{komborobot@googlegroups.com}
\newcommand{\LIPSgrupphemsida}{finns ej}

\newcommand{\LIPSdokumentansvarig}{Mattias Jansson}
\newcommand{\LIPSkund}{ISY, Linköpings universitet, 581\ 83 Linköping}
\newcommand{\LIPSkundkontakt}{Tomas Svensson, 013-281368, tomass@isy.liu.se}
\newcommand{\LIPSkursansvarig}{Tomas Svensson, 013-281368, tomass@isy.liu.se}
\newcommand{\LIPShandledare}{}


\newcommand{\LIPSdokumenttyp}{Efterstudie}
\newcommand{\LIPSredaktor}{Simon Larsson}
\newcommand{\LIPSversion}{0.1}
\newcommand{\LIPSdatum}{\dagensdatum}

% Förslag till versionshantering:
% döp helt enkelt om filen till [dokument]_vX.Y.tex vid varje revision


\newcommand{\LIPSgranskare}{}
\newcommand{\LIPSgranskatdatum}{}
\newcommand{\LIPSgodkannare}{}
\newcommand{\LIPSgodkantdatum}{}

\begin{document}

\LIPStitelsida

%% Argument till \LIPSgruppmedlem: namn, roll i gruppen, telefonnummer, epost
\begin{LIPSprojektidentitet}
  \LIPSgruppmedlem{Simon Larsson}{Projektledare (PL)}{070-7311646}{simla804@student.liu.se}
  \LIPSgruppmedlem{\LIPSdokumentansvarig}{Dokumentansvarig (DOK)}{073-6837074}{matja307@student.liu.se}
  \LIPSgruppmedlem{Gustav Svensk}{Reglersystem (REG)}{073-6208776}{gussv666@student.liu.se}
  \LIPSgruppmedlem{Johan Jönsson}{Mjukvara (KA)}{073-8305758}{johjo939@student.liu.se}
  \LIPSgruppmedlem{Tobias Andersson}{Hårdvara (HV)}{073-7201098}{toban963@student.liu.se}
  \LIPSgruppmedlem{Markus Falck}{Leveransansvarig (LV)}{076-3457552}{marlo265@student.liu.se}
  \LIPSgruppmedlem{Simon Wallin}{Testansvarig (GM)}{076-2300665}{simwa252@student.liu.se}
\end{LIPSprojektidentitet}

\tableofcontents{}
\newpage

%% Argument till \LIPSversionsinfo: versionsnummer, datum, ändringar, utfört av, granskat av
\addcontentsline{toc}{section}{Dokumenthistorik}
\begin{LIPSdokumenthistorik}
\LIPSversionsinfo{0.1}{2012-05-25}{Första utkast.}{matja307}{}
\LIPSversionsinfo{0.1}{2012-05-25}{Stolpar efter diskussion.}{matja307}{}
\end{LIPSdokumenthistorik}
\newpage

\section{Inledning}

\section{Tidsåtgång}
Såg bra ut, tiden räckte lagom. 
\subsection{Arbetsfördelning}
De som hade stora ansvarsområden fick mer att göra, pga att de hade kunskapen att jobba på nya områden. För få som kunde vissa saker.
Ansvarsfördelning kunde ha varit jämnare. 
\subsection{Tidsåtgång jämfört med planerad tid}
Skriv antalet timmar och jämför med planerad. Vissa aktiviteter växte och vissa föll bort. Tester tog mer tid än väntat. Några få aktiviteter kom till. 

\section{Analys av arbete och problem}

\subsection{Vad som hände under de olika faserna}
Före: (Vad, hur?) Kravspec, projektplan.  Svårt att veta hur saker funkar. Ingen erfarenhet. Höftar mycket. 

Under: Designspec. Bygga robot. Kontinuerlig dokumentation hade varit bra. Svårt att skriva, ingen erfarenhet av så stora rapporter. Tajt om tid. 

Efter: Efterstudie. 

\subsection{Hur vi arbetade tillsammans}
Delade upp oss i mindre grupper med en huvudansvarig. Ibland själva. Kunde ha haft bättre kommunikation. (Ex bättre dokumentation) 

Bra med möten en gång i veckan, på bestämda tider. 

\subsection{Hur vi använde projektmallen}
I förarbetet gjorde vi helt enligt mallen. Fungerade jättebra. Slapp uppfinna allt själva. Framför allt designspec och aktivitetsuppdelning. 

Kravspecen borde använts mer, istället för att bara titta på designspecen. Vilka aktiviteter hör till vilket krav? 

Aktiviteter som inte stog med från början tog väldigt lång tid innan de blev gjorda. Vi borde inte tvekat att uppdatera aktivitetslistan när nya aktiviteter komms på. 

\subsection{Hur relationen med beställaren fungerade}
Bra svar på frågor och tydliga kommentarer på inlämnade dokument. Engagerad. Frekvent respons. Tydligt på skillnaden mellan vad som 
är kommentarer från examinator och beställare. 

\subsection{Hur relationen med handledaren fungerade}
Kunde ha utnyttjats mer. Fråga tidigare. Bra svar när vi väl frågade. Tillgänglig. 

\subsection{Tekniska framgångar och problem}

\begin{enumerate}
\item Reglering
Instabilt schassi, bakhjul. 
Bättre tänk innan, fler sensorer.
Många små problem, mycket tid. Minst 60 timmar. 

\item Bakhjul
Hade vi bytt schassi så hade det minskat tiden på motorstyrningen, men det kunde ha gett problem vid ex. linjereglering. 
40 timmar. 

\item Blåtand 
Extern klocka
Mjukvara. 
Dålig data i överföring. Dålig data filtrerades bort. 
Alla problem löstes inte helt, vissa fel filtrerades bort, så bättre lösning kunde ha gjorts. 
Kanske 30 timmar. 
\end{enumerate}

\section{Måluppfyllelse}
\subsection{Vad som har uppnåtts}
Målet var att göra en fungerande robot. Det har vi gjort. Några få ändringar i kravspecifikationen fick göras. 

\subsection{Hur leveransen fungerade}
Bra. Godkända direkt. Tävlingen vanns inte, men banan klarades. Vi kom två. Konkurrenterna kom näst sist. Kraven 
uppfylldes i tid, likaså dokumentationen.  

\subsection{Hur studiesituationen har påverkat projektet}
Vi hade till viss del samma schema, vilket var både bra och dåligt. Bra för möten och liknande, sämre för att vi alla 
kunde sitta samtidigt i Muxen, vilket blev lite trångt runt roboten. 

Projektet tog mycket tid. 

\section{Sammanfattning}
\subsection{De tre viktigaste erfarenheterna}
\begin{itemize}
\item Planera mera! Tänk innan! Flödesschema över systemet.
\item Jobba med hårdvara, inte bara teorin. Läsa datablad. 
\item Versionshanterare för kod och dokumentation. 
\item Kakbak
\end{itemize}

\subsection{Goda råd till dem som ska utföra ett liknande projekt}
\begin{itemize}
\item Planera ännu mera
\item Bättre kommunikation
\item Tydliga krav och mål. Funktionella krav, och inte designmässiga. Spara designen till designspecen. 
\end{itemize}

\end{document}




























