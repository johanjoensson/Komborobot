% !TEX encoding = UTF-8 Unicode

\documentclass[a4paper,12pt]{article}
\usepackage[swedish]{babel}
\usepackage[utf8]{inputenc}
\usepackage{graphicx}
\usepackage{epstopdf}
\usepackage{gensymb}
%% Definitioner för LIPS-dokument

\usepackage[swedish]{babel}
\usepackage[utf8]{inputenc}
\usepackage[T1]{fontenc}
\usepackage{times}
\usepackage{ifthen}

\usepackage[margin=25mm]{geometry}

\usepackage{fancyhdr}
\pagestyle{fancy}
\lhead{}
\chead{\textbf{\LIPSprojekttitel}}
\rhead{\textbf{\textsl{LiTH}}\\\textbf{\LIPSdatum}}
\lfoot{\textbf{\LIPSkursnamn}\\\textbf{\LIPSdokumentansvarig}}
\cfoot{\textbf{\LIPSprojektgrupp}\\\textbf{\LIPSgruppepost}}
\rfoot{\textbf{\textsc{Lip}s}\\\textbf{Sida~\thepage}}

\setlength{\parindent}{0pt}
\setlength{\parskip}{1ex plus 0.5ex minus 0.2ex}


\newcommand{\twodigit}[1]{\ifthenelse{#1<10}{0}{}{#1}}
\newcommand{\dagensdatum}{\number\year-\twodigit{\number\month}-\twodigit{\number\day}}

%% ------------------------------------------
% NYBILD
% Skapar centrerad bild med caption
%   
% #1: Filens url relativt '/bilder/'
% #2:  Caption
% #3: Label
% #4: Skalning
%% ------------------------------------------
\newcommand{\nyBild}[4] 
{\begin{figure}[H]
  \centering
 \includegraphics[angle=0,scale=#4]{bilder/#1}
  \caption{#2}
  \label{fig:#3}
\end{figure}}



%%  Redefinitions of commands containing @
\makeatletter
\makeatother

\newcommand{\LIPStitelsida}{%
{\ }\vspace{45mm}
\begin{center}
  \textbf{\Huge \LIPSdokumenttyp}
\end{center}
\begin{center}
  {\Large Redaktör: \LIPSredaktor}
\end{center}
\begin{center}
  {\Large \textbf{Version \LIPSversion}}
\end{center}
\vfill
\begin{center}
  {\large Status}\\[1.5ex]
  \begin{tabular}{|*{3}{p{40mm}|}}
    \hline
    Granskad & \LIPSgranskare & \LIPSgranskatdatum \\
    \hline
    Godkänd & \LIPSgodkannare & \LIPSgodkantdatum \\
    \hline
  \end{tabular}
\end{center}
\newpage
}


\newenvironment{LIPSprojektidentitet}{%
{\ }\vspace{45mm}
\begin{center}
  {\Large PROJEKTIDENTITET}\\[0.5ex]
  {\small
  \LIPSartaltermin, \LIPSprojektgrupp\\
  Linköpings Tekniska Högskola, ISY
  }
\end{center}
\begin{center}
  {\small Gruppdeltagare}\\
%  \begin{tabular}{|p{30mm}|p{40mm}|p{35mm}|p{45mm}|}
  \begin{tabular}{|l|p{45mm}|p{25mm}|l|}
    \hline
    \textbf{Namn} & \textbf{Ansvar} & \textbf{Telefon} & \textbf{E-post} \\
    \hline
}%
{%
    \hline
  \end{tabular}
\end{center}
\begin{center}
  {\small
    \textbf{E-postlista för hela gruppen}: \LIPSgruppepost\\
    \textbf{Hemsida}: \LIPSgrupphemsida\\[1ex]
    \textbf{Kund}: \LIPSkund\\
    \textbf{Kontaktperson hos kund}: \LIPSkundkontakt\\
    \textbf{Kursansvarig}: \LIPSkursansvarig\\
    \textbf{Handledare}: \LIPShandledare\\
  }
\end{center}
\newpage
}
\newcommand{\LIPSgruppmedlem}[4]{\hline {#1} & {#2} & {#3} & {#4} \\}



\newenvironment{LIPSdokumenthistorik}{%
\begin{center}
  Dokumenthistorik\\[1ex]
  \begin{small}
    \begin{tabular}{|l|l|p{60mm}|l|l|}
      \hline
      \textbf{Version} & \textbf{Datum} & \textbf{Utförda förändringar} & \textbf{Utförda av} & \textbf{Granskad} \\
      }%
    {%
      \hline
    \end{tabular}
  \end{small}
\end{center}
}
\newcommand{\LIPSversionsinfo}[5]{\hline {#1} & {#2} & {#3} & {#4} & {#5} \\}

\newcounter{LIPSkravnummer}
\newcounter{LIPSunderkravnummer}[LIPSkravnummer]
\newenvironment{LIPSkravlista}{%
  \begin{tabular}{|p{25mm}|p{25mm}|p{85mm}|p{5mm}|}
    }%
  {%
    \hline
  \end{tabular}
}
\newcommand{\LIPSkrav}[3]{\hline\stepcounter{LIPSkravnummer}\textbf{Krav nr \arabic{LIPSkravnummer}} & \textbf{{#1}} & {#2} & \textbf{{#3}} \\}
\newcommand{\LIPSunderkrav}[3]{\hline\stepcounter{LIPSunderkravnummer}\textbf{Krav nr \arabic{LIPSkravnummer}\Alph{LIPSunderkravnummer}} & \textbf{{#1}} & {#2} & \textbf{{#3}} \\}





%%% Local Variables: 
%%% mode: latex
%%% TeX-master: "kravspec_mall"
%%% End: 



\newcommand{\LIPSartaltermin}{2012/VT}
\newcommand{\LIPSkursnamn}{TSEA27}

\newcommand{\LIPSprojekttitel}{Komborobot}

\newcommand{\LIPSprojektgrupp}{Grupp 17}
\newcommand{\LIPSgruppepost}{komborobot@googlegroups.com}
\newcommand{\LIPSgrupphemsida}{finns ej}

\newcommand{\LIPSdokumentansvarig}{Mattias Jansson}
\newcommand{\LIPSkund}{ISY, Linköpings universitet, 581\ 83 Linköping}
\newcommand{\LIPSkundkontakt}{Tomas Svensson, 013-281368, tomass@isy.liu.se}
\newcommand{\LIPSkursansvarig}{Tomas Svensson, 013-281368, tomass@isy.liu.se}
\newcommand{\LIPShandledare}{}


\newcommand{\LIPSdokumenttyp}{Efterstudie}
\newcommand{\LIPSredaktor}{Simon Larsson}
\newcommand{\LIPSversion}{0.1}
\newcommand{\LIPSdatum}{\dagensdatum}

% Förslag till versionshantering:
% döp helt enkelt om filen till [dokument]_vX.Y.tex vid varje revision


\newcommand{\LIPSgranskare}{}
\newcommand{\LIPSgranskatdatum}{}
\newcommand{\LIPSgodkannare}{}
\newcommand{\LIPSgodkantdatum}{}

\begin{document}

\LIPStitelsida

%% Argument till \LIPSgruppmedlem: namn, roll i gruppen, telefonnummer, epost
\begin{LIPSprojektidentitet}
  \LIPSgruppmedlem{Simon Larsson}{Projektledare (PL)}{070-7311646}{simla804@student.liu.se}
  \LIPSgruppmedlem{\LIPSdokumentansvarig}{Dokumentansvarig (DOK)}{073-6837074}{matja307@student.liu.se}
  \LIPSgruppmedlem{Gustav Svensk}{Reglersystem (REG)}{073-6208776}{gussv666@student.liu.se}
  \LIPSgruppmedlem{Johan Jönsson}{Mjukvara (KA)}{073-8305758}{johjo939@student.liu.se}
  \LIPSgruppmedlem{Tobias Andersson}{Hårdvara (HV)}{073-7201098}{toban963@student.liu.se}
  \LIPSgruppmedlem{Markus Falck}{Leveransansvarig (LV)}{076-3457552}{marlo265@student.liu.se}
  \LIPSgruppmedlem{Simon Wallin}{Testansvarig (GM)}{076-2300665}{simwa252@student.liu.se}
\end{LIPSprojektidentitet}

\tableofcontents{}
\newpage

%% Argument till \LIPSversionsinfo: versionsnummer, datum, ändringar, utfört av, granskat av
\addcontentsline{toc}{section}{Dokumenthistorik}
\begin{LIPSdokumenthistorik}
\LIPSversionsinfo{0.1}{2012-05-25}{Första utkast.}{matja307}{}
\LIPSversionsinfo{0.1}{2012-05-25}{Stolpar efter diskussion.}{matja307}{}
\end{LIPSdokumenthistorik}
\newpage

\section{Inledning}
Denna studie är en sammanfattning av hur grupp 17, bestående de sju projektmedlemarna vars namn anges på sida 2, arbetat med projektkursen "Elektronikprojekt Y" under vårterminen 2012, samt vilka erfarenheter och lärdomar som kunnat dras av det. Under projektet har gruppen konstruerat en s.k. Komborobot, vilket i detta fall innebär en robot som autonomt kan ta sig fram sig genom en bana bestående dels av en upptejpad linje på golvet (linjen har mörkare färg än underlaget), dels en labyrint uppbyggd av 80 centimeter långa väggmoduler.

\section{Tidsåtgång}
Från det att underfasen av projektet (se avsnitt 3.1) inleddes med att gruppen framställde en designspecifikation för roboten till projektets slutförande fanns en tidsbudget på 140 timmar per gruppmedlem.
\subsection{Tidsåtgång jämfört med planerad tid}
Då gruppen bestod av sju medlemmar skulle totalt 980 timmar användas till under- och efterfasen av projektet. Denna tidsbudget har överskridits något och ser ut att landa på 1020 timmar, vilken vi ändå anser ligga inom rimliga gränser. Tidsplanen som fastställdes i projektets början följdes genom hela arbetets gång. Det som orsakade störst problem med tidsplaneringen var svårigheterna i att förutse hur mycket tid en viss aktivitet skulle ta i anspråk.
\subsection{Arbetsfördelning}
Redan vid projektets start fick varje projektmedlem ett ansvarsområde som de var huvudansvariga över under projektets gång. Ambitionen var att alla skulle hjälpa till så mycket som möjligt under olika områden men det stod ganska snart klart att de med huvudansvar över större områden (t.ex sensor/hårdvara) för det mesta var sysselsatta med sitt område, eftersom det helt enkelt skulle gå åt för mycket tid om alla i gruppen skulle lära sig allt inom de olika områdena på egen hand. De som hade mindre ansvarsområden hade däremot ofta problem att hitta givande arbetsuppgiffter att spendera den budgeterade tiden på eftersom de ofta fick hjälpa till med det som för stunden var mest tidskrävande vilket gjorde det svårt att få in ett "flyt" i arbetet.
För att undvika detta hade det förmodligen varit bättre att sortera upp aktiviteterna i ansvarsområden efter det att tidsplanen fastställts för att får en tidsmässigt jämn fördelning.

\section{Analys av arbete och problem}

\subsection{Vad som hände under de olika faserna}
Före: (Vad, hur?) Kravspec, projektplan.  Svårt att veta hur saker funkar. Ingen erfarenhet. Höftar mycket. 

Under: Designspec. Bygga robot. Kontinuerlig dokumentation hade varit bra. Svårt att skriva, ingen erfarenhet av så stora rapporter. Tajt om tid. 

Efter: Efterstudie. 

\subsection{Hur vi arbetade tillsammans}
Delade upp oss i mindre grupper med en huvudansvarig. Ibland själva. Kunde ha haft bättre kommunikation. (Ex bättre dokumentation) 

Bra med möten en gång i veckan, på bestämda tider. 

\subsection{Hur vi använde projektmallen}
I förarbetet gjorde vi helt enligt mallen. Fungerade jättebra. Slapp uppfinna allt själva. Framför allt designspec och aktivitetsuppdelning. 

Kravspecen borde använts mer, istället för att bara titta på designspecen. Vilka aktiviteter hör till vilket krav? 

Aktiviteter som inte stog med från början tog väldigt lång tid innan de blev gjorda. Vi borde inte tvekat att uppdatera aktivitetslistan när nya aktiviteter komms på. 

\subsection{Hur relationen med beställaren fungerade}
Bra svar på frågor och tydliga kommentarer på inlämnade dokument. Engagerad. Frekvent respons. Tydligt på skillnaden mellan vad som 
är kommentarer från examinator och beställare. 

\subsection{Hur relationen med handledaren fungerade}
Kunde ha utnyttjats mer. Fråga tidigare. Bra svar när vi väl frågade. Tillgänglig. 

\subsection{Tekniska framgångar och problem}

\begin{enumerate}
\item Reglering
Instabilt schassi, bakhjul. 
Bättre tänk innan, fler sensorer.
Många små problem, mycket tid. Minst 60 timmar. 

\item Bakhjul
Hade vi bytt schassi så hade det minskat tiden på motorstyrningen, men det kunde ha gett problem vid ex. linjereglering. 
40 timmar. 

\item Blåtand 
Extern klocka
Mjukvara. 
Dålig data i överföring. Dålig data filtrerades bort. 
Alla problem löstes inte helt, vissa fel filtrerades bort, så bättre lösning kunde ha gjorts. 
Kanske 30 timmar. 
\end{enumerate}

\section{Måluppfyllelse}
Alla krav i kravspecifikationen uppfylldes i tid till leveransen, och roboten klarade banan både vid leveransen samt
under tävlingen. Detta var gruppens primära mål, vilket alltså uppfylldes. 

\subsection{Vad som har uppnåtts}
Målet för gruppen var, förutom att bygga en robot som uppfyllde kraven i kravspecifikationen också att lära oss 
och träna på de kunskaper som krävs för detta. Vi har alla lärt oss mycket, både i projektarbete och de tekniska 
färdigheter som krävdes. 

Roboten uppfyllde i slutändan alla krav i kravspecifikationen, så det målet uppfylldes. Dock så kom vissa förändringar
bland kraven att göras. 

\subsection{Hur leveransen fungerade}
Leveransen fungerade bra, roboten klarade alla krav i kravspecifikationen. Tävlingen vanns inte, men gruppen hade 
egentligen aldrig som primärt mål att vinna, utan snarare att bygga en stabil och fungerande robot.

Alla krav uppfylldes i tid, likaså var all dokumentation färdig innan respektive deadline. 

\subsection{Hur studiesituationen har påverkat projektet}
Vi läste till viss del samma kurser, vilket var både bra och dåligt. Det var bra för möten och liknande, sämre för att vi alla 
kunde sitta samtidigt i Muxen, vilket gjorde att flera behövde använda roboten samtidigt. 

Projektet tog mycket tid från övriga kurser, främst för att den nedlagda tiden inte var jämt utspridd över kursens gång, 
utan var koncentrerad mycket på den andra halvan av terminen. Detta kom sig av dels brist i planeringsarbetet i början
av kursen, men också kursens upplägg. 


\section{Sammanfattning}
\subsection{De tre viktigaste erfarenheterna}
\begin{itemize}
% <<<<<<< HEAD
\item Planera noga vad ni ska göra i projektet, tänk igenom alla steg och
aktiviteter noga! Rita upp ett flödesschema över systemet för att få en
överblick av vad som hänger ihop med vad.
\item Sätt er in i hur hårdvaran faktiskt fungerar, det är jättebra med
teoretisk kunskap men ibland måste man låta sig begränsas av verkligheten.
Många problem är kopplade till hårdvaran och inte teorin. Läs igenom databladen
så ni vet hur saker och ting fungerar. När ni läst igenom databladen, läs igenom
dem igen. 
\item En versionshanterar, t.ex. Git, kan vara användbart att använda sig av för
att förenkla arbetet med kod och dokumentation. Tänk på att alla bör vara
insatta i hur versionshanteraren fungerar. 
\item För att göra era projektmöten trevligare kan det vara en bra ide att någon
bakar något enkelt och bjuder på. Roterande baktjänst är att föredra.
% =======
% \item Den kanske viktigaste erfarenheten ligger i \"före-fasen\", d.v.s. planeringsarbetet. Att planera ett
% så pass start project var nytt för oss alla i gruppen. Det är tydligt att man inte kan lägga ner för mycket 
% tid i början. 
% \item En viktig erfarenhet var att få jobba med hårdvara. Även om vi har gjort laborationer i andra kurser
% så är det en annan sak i detta projekt. Här har vi fått vänja oss vid att inte bara använda hårdvara, utan också
% välja vilka olika komponenter som passar bäst. Även träningen i att läsa datablad har varit nyttig. 
% \item Att använda versionshanterare har varit vitalt för projektets framgång, och träning i detta kommer vara nyttigt
% för framtiden. 
% >>>>>>> 4311263c2876e4d4be479d81ca3b8aac1720ded2
\end{itemize}

\subsection{Goda råd till de som ska utföra ett liknande projekt}
\begin{itemize}
%<<<<<<< HEAD
\item Generellt kan man säga att ni kommer behöva planera mera! Inga undantag.
Lägg några timmar extra på varje dokument när ni tycker att ni är klara, det
kommer behövas. Få med alla aktiviteter ni behöver och inte fler. Se till att
alla vet vad som behöver göras under hela projektets gång.
\item Se till att prata med varandra. Det är inte bra om flera personer väntar
på att någon ska förklara hur något fungerar. Ha möten, där ni uppdaterar
varandra, kontinuerligt så att alla vet hur det ser ut med arbetet. På så sätt
kan ni undvika flaskhalsar.
\item Ha tydliga krav och mål. Om något är oklart i kravspecen måste ni tala med
beställaren för att veta hur det ska tolkas, det är bättre och mer tidseffektivt
att göra rätt direkt. Se till att bara ha funktionell krav i kravspecen, alltså
bara krav på hur roboten ska fungera. Spara robotens design och uppbyggnad till
designspecen, den kan ni vid behov ändra utan att behöva förhandla med
beställaren. 
%=======
%\item Det går alltid att planera och fundera mer under \"före-fasen\". Tänk igenom allt noggrant från början
%så blir det mycket lättare senare. 
%\item Var noga med att kommunicera inom gruppen, det är viktigt att alla har bra koll på vad som händer, 
%speciellt om något ändras från hur det var tänkt från början. 
%\item Tänk noga igenom kraven i kravspecifikationen. De bör vara tydliga och konkreta. Ta bara med det 
%som verkligen behöver finnas med i kravspecifikationen, spara designidéerna till designspecifikationen. 
%>>>>>>> 4311263c2876e4d4be479d81ca3b8aac1720ded2
\end{itemize}

\end{document}




























