\section{PC mjukvara}
PC-mjukvaran har som uppgift att låta användaren kommunicera med roboten via ett
enkelt gränssnitt. Via PC-mjukvaran kan användaren få intressant information
från robotens olika moduler, t.ex avstånd till väggar eller vilket styrkommando
som nu utförs. PC-mjukvaran kommunicerar med roboten via blåtand.

\subsection{Implementation}

Mjukvaran är skriven i programspråket C och använder utöver Cs standardbibliotek
även gränssnittet BlueZ för att kommunicera via blåtand, biblioteket SDL för att
generera kommandon till roboten via tangentbordet samt NCurses för att
åskådliggöra information i terminalen på ett trevligt och överskådligt sätt..

Mjukvaran består av 2 huvuddelar, input\_control samt send\_receive. input\_control
hanterar knapptryckningar från användaren och genererar instruktioner att skicka
till roboten, intruktionen placeras sedan i en enkel databas (instr\_db).
send\_receive i sin tur läser in instruktionen från databasen och om användaren
har genererat en ny instruktion kommer denna skickas till roboten via blåtand
(instruktioner som redan skcikat till roboten kommer alltså inte att skickas
 igen), ifall instruktionen innehåller information om önskad hastighet på
roboten eller trimnivåer på motorerna så kommer detta också visas på skärmen.
När instruktionen skickats till roboten kommer send_receive invänta 2 databyte
från roboten (vilket kan vara sensorinfo, specialkommando el. dyl.) som sedan
kommer åskådligjöras på skärmen.
\subsection{Användande}

För att börja använda roboten startar man först programmet input_control, detta
initierar databasen instr\_db och ger användaren möjlighet att skapa
styrkommandon åt roboten med hjälp av tangentbordstryckningar. Sedan startar man
programmet send_receive som ansluter till roboten via blåtand och börjar skicka
samt ta emot data från roboten.

Då roboten befinner sig i fjärrstyrt läge används följande tangentbindningar för
att generera styrkommandon:
w - Kör framåt
s - Kör bakåt
a - rotera vänster
d - rotera höger
q - vänstersväng (mjuk kurva)
e - högersväng (mjuk kurva)
mellanslag - stanna roboten
pil upp - öka hastigheten, ingen effekt tills man skickar ett kommando som
utnyttjar hastigheten
pil ner - sänk hastigheten, ingen effekt tills man skickar ett kommando som
utnyttjar hastigheten.
pil höger - trim höger, öka effekten på höger motor en aning (små steg,
		användbart för finjustering)
pil vänster - trim vänster, öka effekten på vänster motor en aning (små steg,
		användbart för finjustering)
o - nollställ trim
esc - avsluta input\_control

För att avsluta send\_receive används (ctr + c) vilket kommer återställa
terminalen åt användaren.

% \subsection{Implementation}
% 
% Mjukvaran är skriven i programspråken C samt Tcl och använder utöver Cs standardbibliotek
% även gränssnittet BlueZ för att kommunicera via blåtand.
% 
% \subsection{Användande}
% 
% Mjukvaran kommer utgöras av ett fönster där användaren kan se information
% skickad från robotens sensorer och dess styrenhet samt skicka styrkommandon till
% roboten.
% 
% I fjärrstyrt läge kan användaren välja att använda de pilknappar som finns i
% fönstret för att styra roboten eller så kan användaren använda piltangenterna på
% tangentbordet.
% 
% I autonomt läge kommer gränssnittet visa information från robotens sensorer samt
% vilket styrkommando som roboten just nu utför.
% 
% Gränssnittet är enkelt att använda, kommandona är logiska och simpla och
% informationen från roboten visas på ett tydligt och lättförståeligt sätt.
% 
