% !TEX encoding = UTF-8 Unicode

%% --------------------------------------------------------------------------------------------------------------------------------------------
% Protokoll - En beskrivning av de protokoll som används för kommunikationen, och vad de är bra för.
%
%
% --marlo265, 2012-05-06
%% --------------------------------------------------------------------------------------------------------------------------------------------
 
\section{Protokoll}
\label{sec:protokoll}

För att förenkla kommunikation mellan enheter så används ett enhetligt
protokoll, detta gör det också väldigt mycket lättare att lägga till fler
enheter till roboten.


\subsection{Generell överföring mellan enheter}
Kommunikationen mellan de olika enheterna sker via en SPI-buss (se 
\ref{sec:SPI}).
All kommunikation består av en headerbyte följt av en databyte.
I headerbyten anges till vilken enhet den efterföljande informationen ska, 
samt hur den ska tolkas av den mottagande enheten.
De tre mest signifikanta bitarna i headern anger till vilken enhet 
informationen ska och de 5 minst signifikanta bitarna anger 
hur denna information ska behandlas. Uppdelningen kan ses i tabell 
\ref{tab:header}.  Kodningen av de tre mest signifikanta bitarna kan ses i 
tabell \ref{tab:headerkod}.

All kommunikation går via kommunikationsenheten, så om sensorenheten vill 
skicka data till styrenheten skickar den en headerbyte med styrenhetens adress 
och en databyte till kommunikationsenheten som 
tolkar headern och skickar headern och databyten till styrenheten.
\begin{table}[h]
  \centering
  \begin{tabular}{| c | c | c | c | c | c |}
    \hline
    Målenhet [3 bitar] & A & B & C & D & E \\
    \hline
  \end{tabular}
  \caption{Bituppdelning av header}
  \label{tab:header}
\end{table}

\begin{table}[h]
  \centering
  \begin{tabular}{l l}
    \textbf{3 MSB} & \textbf{Målenhet} \\
    000 & Kommunikationsenhet \\
    001 & Sensorenhet \\
    010 & Styrenhet \\
    100 & PC \\
    110 & PC och Styrenhet\\
  \end{tabular}
  \caption{Kodning av header}
  \label{tab:headerkod}
\end{table}


\subsection{PC-mjukvara $\longleftrightarrow$ Styrenhet}
Då styr- eller sensorenheten sänder data till PC-mjukvaran kommer mjukvaran att
översätta header-byten samt data-byten så att man enkelt kan åskådliggöra
instruktionen som sändes. 
PC-mjukvaran skickar styrkommandon till roboten då roboten är i fjärrstyrt läge.
Bituppdelningen av styrkommandon kan ses i tabell \ref{tab:styrbitar} och
kodningen kan ses i tabell \ref{tab:styrkommando}.

Om styrenheten får en header den inte känner igen så skickar den ett
felmeddelande till PCn.

\begin{table}[h] 
  \centering
  \begin{tabular}{| c | c |}
    \hline
    Kommando [4 bitar] & Hastighet [4 bitar] \\ \hline
  \end{tabular}
  \caption{Bituppdelning av styrkommandon}
  \label{tab:styrbitar}
\end{table}

\begin{table}[h] 
  \centering
  \begin{tabular}{l l}
    \textbf{4 MSB} & \textbf{Kommando} \\
    0000 & Stopp \\
    0001 & Framåt \\
    0010 & Fram vänster \\
    0011 & Fram höger \\
    0100 & Rotera vänster \\
    0101 & Rotera höger \\
    0110 & Back \\
    0111 & Set höger \\
    1000 & Set vänster \\
    1001 & Trim vänster \\
    1010 & Trim höger \\
    1011 & No trim \\
  \end{tabular}
  \caption{Kommandokodning}
  \label{tab:styrkommando}
\end{table}


\subsection{Sensorenhet $\longleftrightarrow$ Styrenhet, PC-mjukvara}
När sensorenheten skickar information till styrenheten och PCn så anger
E-flaggan om roboten är i autonomt eller fjärrstyrt läge.
D-flaggan avgör om databyten som mottagits är ett specialkommando eller inte.
A-flaggan anger om roboten kör i labyrint eller om den följer en linje.
Flaggornas lägen kan ses i tabell \ref{tab:adeflaggor}.

Efterföljande data är antingen ett specialkommando,
se tabell \ref{tab:special}, avstånd till väggar eller avstånd till 
centrumlinjen. Databytens bituppdelning vid specialkommando kan ses i tabell
\ref{tab:specialbitar}. Avstånd till centrumlinjen skickas bara till styrenheten.
B- och C-flaggorna anger vilken avståndssensor som skickas, kodningen kan ses i
tabell \ref{tab:bcflaggor}. Om både B- och C-flaggan är satt kan det vara en av
tre sensorer, då avgör de översta databitarna vilken sensor det är, se tabell
\ref{tab:kortsensor}.

PC-mjukvaran skickar data till sensorenheten då tröskelvärdet på linjesensorerna
kalibreras, se \ref{sec:linjesensor}.

\begin{table}[H]
  \centering
  \begin{tabular}{l l l l l l l l}
    \textbf{A} & \textbf{Läge} & & \textbf{D} & \textbf{Typ av data} & & \textbf{E} & \textbf{Läge} \\
    0 & Labyrint & & 0 & Sensordata & & 0 & Fjärrstyrt \\
    1 & Linje & & 1 & Specialkommando & & 1 & Autonomt \\
  \end{tabular}
  \caption{A-,D- och E-flaggor}
  \label{tab:adeflaggor}
\end{table}


\begin{table}[H]
  \centering
  \begin{tabular}{l l}
    \textbf{BC} & \textbf{Sensor} \\
    00 & Fram vänster \\
    01 & Bak vänster \\
    10 & Fram höger\\
    11 & Bak höger, Kort vänster, Kort höger\\
  \end{tabular}
  \caption{B- och C-flaggor}
  \label{tab:bcflaggor}
\end{table}


\begin{table}[H]
  \centering
  \begin{tabular}{l l}
    \textbf{2 MSB i databyte} & \textbf{Sensor} \\
    0- & Bak höger \\
    10 & Kort höger \\
    11 & Kort vänster\\
  \end{tabular}
  \caption{B- och C-flaggor}
  \label{tab:kortsensor}
\end{table}

\begin{table}[h] 
  \centering
  \begin{tabular}{| c | c | c | c | c | c |}
    \hline
     0 & Specialkommando [3 bitar] & 0 & 0 & 0 & 0 \\ \hline
  \end{tabular}
  \caption{Bituppdelning av specialkommandon}
  \label{tab:specialbitar}
\end{table}

\begin{table}[h]
  \centering
  \begin{tabular}{l l}
    \textbf{Kodning} & \textbf{Specialkommando} \\
    000 & Återuppta vanlig reglering\\
    001 & Kör rakt fram \\
    010 & Sväng höger 90\degree \\
    011 & Sväng vänster 90\degree \\
    100 & Rotera höger 90\degree \\
    110 & Rotera vänster 90\degree \\
  \end{tabular}
  \caption{Kodning av specialkommandon}
  \label{tab:special}
\end{table}
