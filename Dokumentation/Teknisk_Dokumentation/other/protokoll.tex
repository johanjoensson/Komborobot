\section{Protokoll}
Protokollen definieras i två lager, ett som enbart innehåller packetstorleken 
på underliggande mer beskrivande lager. 
\subsection{Modulgenerella format}
Det enda modulgenerella formatet som används är dataformatet för sensordatan.
Eftersom de optiska avståndsmätarna ger ett 10 bitars värde så skikas två byte
med de 6 största bitarna nollade.

\begin{table}[H]
	\caption{Dataformat för sensordata}
	\begin{center}
	\begin{tabular}{| c | c | c | c | c | c | c | c |} \hline
		0 & 0 & 0 & 0 & 0 & 0 &
		MSBs sensordata [2 bitar] & LSBs sensordata  [1 byte] \\ \hline
	\end{tabular}
	\end{center}
\end{table}

\subsection{Mellan PC-mjukvara och kartmodulen}
PC-mjukvaran och kartmodulen använder en protokollstack där paketdatan wrapas
med paketlängde för att sendan skickas som en paketlängd på 1 byte och
sedan så många bytes som paketdatan.

\begin{table}[H]
	\caption{Protokollstack för kommunkation mellan PC-mjukvaran
	och kartmodulen}
	\begin{center}
	\begin{tabular}{| c |} \hline
		Paketdata \\ \hline
		Paketlängd \\ \hline
	\end{tabular}
	\end{center}
\end{table}

\begin{table}[H]
	\caption{Dataformat för adresseringslagret}
	\begin{center}
	\begin{tabular}{| c | c |} \hline
		Paketlängd [1 byte] & Data payload [0-255 byte] \\ \hline
	\end{tabular}
	\end{center}
\end{table}

\subsubsection{PC-mjukvara $\rightarrow$ kartmodulen}
När PC-mjukvaran skickar till kartmodulen skickas det först en byte med ett
kommando och sedan inget eller flera bytes med data tillhörande det kommandot
om sådan data behövs.

\begin{table}[H]
	\caption{Dataformat för kommunikation från PC-mjukvaran till
	kartmodulen}
	\begin{center}
	\begin{tabular}{| c | c |} \hline
		Kommando [1 byte] & Kommandospecifik data [0-254 byte] \\ \hline
	\end{tabular}
	\end{center}
\end{table}

\begin{table}[H]
	\caption{Kommandon till kartmodulen från PC-mjukvaran}
	\begin{tabular}{l l}
		\textbf{Kommando} & \textbf{Innebörd} \\
		0x0	& 	Reset \\
		0x1	&	Stop \\
		0x2 	&	Begäran om karta \\
		0x3	&	Fjärrstyrningskommando \\
		0x4	&	Debugdataförfrågan \\
	\end{tabular}
\end{table}

\begin{table}[H]
	\caption{Dataformat för fjärrstyrningskommandon}
	\begin{center}
	\begin{tabular}{| c | c |} \hline
		Framåt / Bakåt [1 byte] & Vänster / Höger [1 byte] \\ \hline
	\end{tabular}
	\end{center}
\end{table}

\begin{table}[H]
	\caption{Fjärrstyrningskommando}
	\begin{tabular}{l l}
		\textbf{Kommando} & \textbf{Innebörd} \\
		0x0	&	Stop (NOP) \\
		0x1	&	Framåt / Vänster \\
		0x2	&	Bakåt / Höger \\
	\end{tabular}
\end{table}

\subsubsection{Kartmodul $\rightarrow$ PC-mjukvara}
Vid förfrågan om debugdata så kommer allt tillgänglig debugdata att skickas.

För att få roboten att röra sig så skickar man ett kommando. Roboten kommer att
utföra det kommandot fram till ett annat kommando skickas.

\begin{table}[H]
	\caption{Dataformat för kommunikation från kartmodulen till
	PC-mjukvaran}
	\begin{center}
	\begin{tabular}{| c | c |} \hline
		Datatyp [1 byte] & Data [0-254 byte] \\ \hline
	\end{tabular}
	\end{center}
\end{table}

\begin{table}[H]
	\caption{Data till PC-mjukvaran från kartmodulen}
	\begin{tabular}{l l}
		\textbf{Kommando} & \textbf{Innebörd} \\
		0x0	& 	Kartdata \\
		0x1	&	Sensordata, avstånd fram \\
		0x2	&	Sensordata, avstånd vänster \\
		0x3	&	Sensordata, avstånd höger \\
		0x4 	&	Sensordata, fart vänster \\
		0x5	&	Sensordata, fart höger \\
		0x6	&	Sensordata, gyro \\
		0x7	&	Tillståndsbeskrivning \\
	\end{tabular}
\end{table}
%GNAA

Kartdatat skickas som 5 $*$ 17 byte där en byte innehåller fyra kartdelar,
förutom den sista i varje rad där bara en kartdel lagras. 5 byte innehåller då
alltså 17 kartdelar och det skickas därmed en karta med 17 $*$ 17 kartdelar.
En rad (tabell \ref{fig:datakart}) skickas 17 ggr.

\begin{table}[H]
	\caption{Dataformat för kartdatan (en rad)}
	\label{fig:datakart}
	\begin{center}
	\begin{tabular}{| c | c | c | c |} \hline
		Kartdel [2 bit] & Kartdel [2 bit] & Kartdel [2 bit] &
		Kartdel [2 bit] \\ \hline
		Kartdel [2 bit] & Kartdel [2 bit] & Kartdel [2 bit] &
		Kartdel [2 bit] \\ \hline
		Kartdel [2 bit] & Kartdel [2 bit] & Kartdel [2 bit] &
		Kartdel [2 bit] \\ \hline
		Kartdel [2 bit] & Kartdel [2 bit] & Kartdel [2 bit] &
		Kartdel [2 bit] \\ \hline
		\multicolumn{3}{| c |}{Reserverad [6 bit]} &
		Kartdel [2 bit] \\ \hline
	\end{tabular}
	\end{center}
\end{table}

\begin{table}[H]
	\caption{Kartdelspecifikation}
	\begin{tabular}{l l}
		\textbf{Värde} & \textbf{Innebörd} \\
		0x0	&	Okänd \\
		0x1	&	Vägg \\
		0x2	&	Körbart \\
		0x3 	&	Roboten \\
	\end{tabular}
\end{table}

Tillståndsbeskrivningen är en ASCII-kodad sträng med maxlängd 254 tecken.

\subsection{Internt kartmodulsprotokoll}
Robotens lägen är i kartmodulen uppdelade i två skilda program. Dessa
lägesprogram kommunicerar med ett gemensamt masterprogram (som hanterar I/O)
enligt nedanstående ASCII-protokoll.

\subsubsection{Master $\rightarrow$ läge}

\begin{table}[H]
	\caption{Kommandon till lägesprogram från masterprogram}
	\begin{tabular}{l l l}
		\textbf{Kommando} & \textbf{Innebörd} & \textbf{Argument}\\
		CONTEST	&	Karttävlingssignal	& - \\
		MAP	&	Kartförfrågan		& - \\
		SENSOR	&	Sensorvärde		& ID (heltal [0, 5]),
		Värde (heltal [0, 65535])\\
		STATUS	&	Statusförfrågan		& - \\
		STEER	&	Styrförfrågan från PC	& Vänster
		(heltal [-128, 127]), Höger (heltal [-128, 127])
	\end{tabular}
\end{table}

\subsubsection{Läge $\rightarrow$ master}

\begin{table}[H]
	\caption{Kommandon till masterprogram från lägesprogram}
	\begin{tabular}{l l l}
		\textbf{Kommando} & \textbf{Innebörd} & \textbf{Argument} \\
		MAP	&	Kartinformation		& - \\
		STATUS	&	Statusinformation	& Informationssträng 
		(0 - 255 tecken) \\
		STEER	&	Styrkommando		& Vänster
		(heltal [-128, 127]), Höger (heltal [-128, 127]) \\
	\end{tabular}
\end{table}

\subsection{Kartmodul $\rightarrow$ styrmodul}
Från kartmodulen till styrmodulen skickas två signed bytes, en för hur mycket 
var sida ska PWM:as. Vid negativt värde backar man. Om båda fälten är 0
räknas det som NOP.

\begin{table}[H]
	\caption{Dataformat för styrningen}
	\begin{center}
	\begin{tabular}{| c | c |} \hline
		Vänster sida [1 byte] &
		Höger sida [1 byte] \\ \hline
	\end{tabular}
	\end{center}
\end{table}

\subsection{Kartmodul $\leftrightarrow$ sensormodul}
Data som ska skickas från kartmodulen till sensormodulen:

\begin{table}[H]
	\caption{Bitmaskning för val av sensor som ska pollas}
	\begin{center}
	\begin{tabular}{| c | c | c | c | c | c | c | c |} \hline
		0 & 0 &
		Gyro &
		Fart Höger & Fart Vänster &
		Avstånd Höger & Avstånd Vänster & Avstånd Fram \\ \hline
	\end{tabular}
	\end{center}
\end{table}

Data som ska skickas från sensormodulen till kartmodulen
\begin{table}[H]
	\caption{Dataformat för sändning av sensordata till kartmodulen}
	\begin{center}
	\begin{tabular}{| c | c |} \hline
		Vilken sensor [1 byte] &
		Sensordata [2 byte] \\ \hline
	\end{tabular}
	\end{center}
\end{table}

\begin{table}[H]
	\caption{Vilken sensor vars data som skickas}
	\begin{tabular}{l l}
		\textbf{Värde} & \textbf{Vilken sensor} \\
		0x0	&	Avstånd Fram \\
		0x1	&	Avstånd Vänster \\
		0x2	&	Avstånd Höger \\
		0x3	&	Fart Vänster \\
		0x4	&	Fart Höger \\
		0x5	&	Gyro \\
	\end{tabular}
\end{table}

\subsection{Gränssnitt mellan reglersystem samt kartalgoritm}
Reglersystemet och kartalgoritmen kommunicerar enligt ett enkelt
stateprotokoll. Kartalgoritmen skickar styrkommandon till reglersystemet. Dessa
kommandon ligger på en väldigt hög abstraktionsnivå (se tabell \ref{ai_reg}),
och involverar framåtförflyttning samt 90-graders rotationer.

När ett styrkommando är utfört, antingen helt eller till den del reglersystemet
tillåter roboten att förflytta sig, så skickar reglersystemet ett
DONE-meddelande, som innehåller robotens relativa position sedan senaste
förflyttningen. För att avbryta ett styrkommando kan STOP användas (se tabell
\ref{ai_reg}). Reglerloopen stannar då roboten samt skickar tillbaka ett
DONE-meddelande.

Endast ett styrkommando kan hanteras i taget. För att byta styrkommando när
man redan utför ett styrkommando så måste först ett STOP skickas. För att 
hålla roboten rätvinklig mot kartans väggar så kan ej rotationskommandon
avbrytas.

När något har hänt, så som att en vägg har upptäckts eller försvunnit så
meddelar reglersystemet detta till kartalgortimen (genom ett EVENT-meddelande).
Kartalgoritmen kan baserat på dessa händelser ändra robotens rörelser.

Meddelanden är insvepta i paranteser. Kommandot och varje argument är
skilda med ett mellanslag, och varje meddelande avslutas med newline.

Reglersystemet agerar även mellanlager mellan masterprogrammet och
kartalgoritmen. CONTEST samt MAP skickas genom reglersystemet.

\begin{table}[H]
	\caption{AI $\rightarrow$ Reglersystem}
	\label{ai_reg}
	\begin{tabularx}{\linewidth}{l l X}
	\textbf{Kommando} & \textbf{Argument} & \textbf{Beskrivning} \\
	FWD	& $n$ (heltal > 0) & Åk framåt $n$ cm \\

	LEFT	& $n$ (heltal > 0)& Följ vänster vägg $n$ centimeter, kommer att
	avslutas när det inte finns en vägg att följa. \\

	MAP	& 289 tecken & Argumenten representerar rad-för-rad en 17x17
	stor karta (se tabell \ref{map_c}). \\

	RIGHT	& $n$ (heltal > 0)& Som LEFT, fast för högervägg. \\

	ROT	& $n$ (heltal)	& Rotera $n * 90$ grader, $n < 0$ ger
	vänsterrotation, och $n > 0$ högerrotation\\

	STOP	& -		& Påtvinga avslut av styrbeslut, reglersystemet
	kommer att skicka tillbaka avslutsinfo (DONE). Rotationer kan inte
	avbrytas. \\
	\end{tabularx}
\end{table}

\begin{table}[H]
	\caption{Reglersystem $\rightarrow$ AI}
	\label{reg_ai}
	\begin{tabularx}{\linewidth}{l l X}
	\textbf{Kommando} & \textbf{Argument} & \textbf{Beskrivning} \\
	CONTEST	& ($l$ $r$ $f$) & Tävlingsstartsmeddelande. $l$, $r$ och $f$ är
	avståndsvärden för vänster, höger samt framåt (i cm), och de ligger
	inslutna i paranteser.  \\

	DONE	& $\Delta a$, ($l$ $r$ $f$) & Indikerar att ett
	styrkommando är slutfört. $\Delta a$ är ett heltal som visar hur långt
	roboten har åkt framåt jämfört med var den stod innan (uttryckt i cm).
	$l$, $r$ och $f$ är avståndsvärden för vänster, höger samt framåt
	(i cm), och de ligger inslutna i paranteser. \\

	EVENT	& ($l$ $r$ $f$) & Händelse som visar att något har hänt
	i robotens omgivning. $l$, $r$ och $f$ är avståndsvärden för vänster,
	höger samt framåt (i cm), och de ligger inslutna i parenteser. \\

	MAP	&	-	& Kartförfrågan \\
	\end{tabularx}
\end{table}

\begin{table}[H]
	\caption{Karttecken för kommunikation mellan AI och reglersystem}
	\label{map_c}
	\begin{tabular}{l l}
	\textbf{Tecken}	& \textbf{Beskrivning} \\
	$C$	& Fri tile \\
	$W$	& Vägg \\
	$R$	& Robot \\
	$U$	& Okänt \\
	\end{tabular}
\end{table}
