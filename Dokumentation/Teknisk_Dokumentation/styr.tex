\section{Styrenhet}

Styrenhetens huvudsakliga uppgift är att ta emot data från
kommunikationsenheten och omvandla det till PWM signaler som styr motorerna.
Styrningen av motorerna sker antingen autonomt med reglering eller 
manuellt genom styrkommandon.

\subsection{Hårdvara}

Styrenheten består av en AVR Atmega16 som sänder styrsignalerna till motorerna, 
En brytknapp som väljer mellan autonomt och manuellt läge, en startknapp och en 
resetknapp.  

\begin{figure}[H]
  \centering
 \includegraphics[angle=0,scale=0.5]{bilder/PIN_styr.jpg}
  \caption{Styrenhetens pin-anslutningar}
  \label{fig:PINstyr}
\end{figure}


\subsection{Mjukvara}

Mjukvaran i styrenheten är indelad i två delar, en manuell del och en autonom del.
I den manuella delen styrs motorerna via instruktioner från PCn som skickas 
genom kommunikationsenheten. PCn skickar då direkta kommandon om vad 
den ska göra. I den autonoma delen kommer styrenheten istället ta emot 
sensordata, ursprungligen skickat från sensorenheten. Styrenheten sköter sedan 
regleringen utifrån dessa sensorvärden. I båda fallen ser styrenheten till att sätta
hastigheten och riktningen på motorerna.

\subsubsection{Autonomt läge}

Styrenhetens uppgift i autonomt läge är kortfattat att bestämma vilken vinkelhastighet de båda hjulen ska rotera i. 
För att ta dessa beslut tar styrenheten mot en stor mängd data från sensorenheten som analyseras.
För att styrenheten ska kunna veta vilken typ av data som mottas märks detta upp i den
medföljande headern, se \ref{sec:protokoll}. 

Datan används sedan olika beroende på vad headern innehåller för information:
 
 \begin{itemize}
 
\item Innehåller datan information från linjeföljning, och roboten befinner sig i linjeläge, 
så anropas linjeregulatorn, se \ref{sec:linjereglering}.

\item Innehåller datan information om avstånd till väggar, och roboten är i labyrintläge,
 så anropas labyrintregulatorn, se \ref{sec:labyrintreglering}.

\item Innehåller datan koden till något typ av så kallat specialkommando, t.ex, en 90\degree 
sväng så utförs en förprogrammerad serie av styrbeslut med tillhörande hastigheter och tider.

\item Innehåller datan koden till stoppkommandot så slutar styrenheten att motta data via databussen.

\item Känner sensorenheten inte igen den header som skickats, kanske pågrund av kommunikationsfel,
så kastas datan och en felmeddelande till datorn.

\end{itemize}

Styrenheten sätts igång med en extern knapp på roboten, vilket aktiverar enheten som då börjar
reglera efter den data som skickas till den.

\subsubsection{Manuell styrning}

I manuellt läge styrs roboten med styrsignaler som skickas från PCn via 
kommunikationsenheten. I mjukvaran liger kod för styrkommandona:
\begin{itemize}
\item Kör framåt - sätter hastighet och riktning framåt
\item Kör bakåt - sätter hastigheten och riktning bakåt
\item Sväng höger - svänger höger snett framåt
\item Sväng vänster - svänger vänster snett framåt
\item Rotera höger - Roterar på platsen åt höger
\item Rotera vänster - Roterar på platsen åt vänster
\item Stopp - Motorerna stannar
\end{itemize}
Styrkommandona anropas med inkommandot speed som ger motorerna den
önskade hastigheten. Hastigheten på respektive motor kommer att skalas i en subrutin
där även trim läggs till. Detta så att trim kommer att löpa över alla styrkommandon.

\subsubsection{Trimning}
Trim är en funktion trim som ökar respektive sänker hastigheten på ett hjul. 
Detta så att man ska kunna öka precisionen vid manuell styrning. Trimning 
anropas från PCn och behöver ingen inparameter.

\subsection{Reglering}
Två olika regleralgoritmer används i roboten, en när roboten följer linjer och
en när roboten kör i labyrinter.
\subsubsection{Linjereglering}
\label{sec:linjereglering}
Utöver att uppfatta var på linjen roboten för närvarande befinner sig är 
linjeregleringen är anpassad för att uppfatta i vilken riktning roboten för 
närvarande färdas. Datan som styrenheten tar emot när roboten befinner sig i 
linjeläge innehåller en kod som visar vilken sensor som är närmast mitten på 
tejpen. Är 2 sensorer för närvarande på tejpen väljs den längst åt höger. 
Koden kan vara något av 11 diskreta värden som sträcker sig från -127 till 127.

Då detta värde ändras så bestäms i vilken riktning relativt tejpen som 
roboten för närvarande rör sig. Detta avgör sedan hur roboten ska bete sig.

Kommer roboten in högerifrån, och tejpen ligger över de högra sensorerna 
utförs en en vänstersväng som blir kraftigare ju närmare de vänstra 
sensorerna roboten kommer. Utförandet är detsamma, fast motriktat, om roboten 
kommer in vänsterifrån, och tejpen ligger över de vänstra sensorerna.

Kommer roboten in vänsterifrån, och tejpen ligger över de vänstra sensorerna 
så utförs en kraftig sväng som avtar när tejpen närmar sig de högra sensorerna
.Utförandet är detsamma, fast motriktat, om roboten kommer in högerifrån, och 
tejpen ligger över de högra sensorerna.


Linjeregulatorn har utöver detta flera olika egenskaper för att förbättra 
regleringen:
\begin{itemize}
\item Visar linjesensorerna att samma sensor varit mitt över tejpen i mer än 8st 
avläsningar, så utgår regulatorn ifrån att tejpen för tillfället är rak och roboten 
beordras att röra sig rakt framåt. Under de första 8 avläsningarna som samma 
sensor varit i mitten av tejpen så utgår regulatorn att roboten har samma riktning 
relativt tejpen som tidigare.
\item Visar linjesensorerna att att mitten av tejpen rört sig än 2 sensorer eller mer
 under de senaste 20 avläsningar, så dubblas regulatorns "effekt".
\item Om roboten hamnat utanför tejpen, det vill säga att den av någon anledning 
kört av, så svänger regulatorn kraftigt åt den riktning där tejpen senast uppfattats
\end{itemize}



\subsubsection{Labyrintreglering}
\label{sec:labyrintreglering}
Labyrintregleringen består av 3 olika delar med olika vikter,
en del som ser till att roboten går rakt (P-del), en del som ser till att
roboten håller sig i mitten av labyrinten (M-del) och en del som motverkar
svängningar (D-del).


P-delen använder sig av sidosensorerna för att hålla roboten parallell med
labyrintväggen. En vägg följs i taget, och kommer roboten för nära en vägg så
följs istället den andra väggen. Detta är den huvudsakliga regleringen och har
hög vikt.


M-delen använder sig av frontsensorerna för att hålla sig i mitten av
labyrinten, den försöker att hålla skillnaden mellan höger- och vänstersensorn
till 0. Om det bara finns en vägg att reglera på avaktiveras denna del. Då det
inte är helt avgörande om roboten är i mitten eller lite åt sidan i labyrinten
har denna del låg vikt.


D-delen ser till att roboten rör sig lugnt och inte börjar att oscillera i
labyrinten. Detta gör den genom att kika på skillnaden på sidosensorerna och är
den skillnaden stor så motverkar den P-delen. Eftersom D-delen reglerar på
relativt små skillnader har den hög vikt.
\label{reglering}

\subsubsection{Specialkommandon}
För att utföra till exempel köra rakt framåt eller utföra 90\degree svängar använder
roboten något som kallas för specialkommandon. Dessa är förprogrammerade 
sekvenser av motorhastigheter, som utförs när styrenheten mottar vissa speciella
headerfilen. Den flagga i headern som avgör om det skickade kommandot är ett
specialkommando är E-flaggan, se \ref{protokoll}.

Roboten har sex olika specialkommandon:

 \begin{itemize}
\item Återuppta vanlig reglering
\end{itemize}
Specialkommandot används inte för närvarande, men det finns programmerat ett
kommando som tvingar styrenheten att omedelbart avsluta pågående specialkommando
och återuppta vanlig reglering.

\begin{itemize}
\item Kör rakt fram
\end{itemize}
Specialkommandot kör rakt fram används för att köra framåt i en korsning. Tiden, och hastigheten,
 som motorerna får resulterar i att roboten kör framåt  i ungefär 100 cm, vilket är sträckan mellan 
 början av en korsning fram tills det att hela roboten står mellan väggarna på den labyrint som är 
 på andra sidan.
 
 \begin{itemize}
\item Sväng höger/vänster 90\degree (Två specialkommandon) 
\end{itemize}
Specialkommandot används för att utföra 90\degree  svängar i korsningar. Utförandet som båbörjas
då styrenheten mottar detta kommando är att roboten först kör rakt framåt i cirka 40 cm och därefter
gör en 90\degree sväng i den givna riktningen. Därefter kör roboten framåt i ungefär 50 cm för 
att komma in mellan de väggar som finns på sidan av korsningen som.

 \begin{itemize}
\item Rotera höger/vänster 90\degree (Två specialkommandon) 
\end{itemize}
Specialkommandot utförs då roboten upptäckt en 90\degree  kurva labyrinten. Roboten utför efter 
mottaget kommando en 90\degree sväng åt den givna riktningen, utan körning framåt varken innan 
eller efter svängen.




 

\subsubsection{Manuell läge}

I manuellt läge styrs Roboten med styrsignaler som skickas från PCn via 
kommunikationsenheten. I mjukvaran liger kod för kommandona; rakt 
fram, bakåt, sväng höger, sväng vänster, rotera höger, rotera vänster
och stopp. Det finns även en funktion trim som ökar respektive sänker 
hastigheten på ett hjul. Detta så att man ska kunna öka precisionen vid 
rak körning. Styrkommandona, som kommer in via SPI bussen, tolkas 
tolkas först och anropar sedan den funktion som efterfrågades av PCn



	

