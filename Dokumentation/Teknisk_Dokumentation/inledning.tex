% !TEX encoding = UTF-8 Unicode

%% --------------------------------------------------------------------------------------------------------------------------------------------
% INLEDNING - Bakgrund och syfte
%
% - Bakgrund
% - Syfte
%
% --mj 20120506
%% --------------------------------------------------------------------------------------------------------------------------------------------


\section{Inledning}
Detta dokument har till syfte att beskriva uppbyggnaden av Komborobot.
Dokumentet innehåller dels en beskrivning av den hårdvara som roboten är 
uppbyggd med, samt mjukvaran som styr den. I den tekniska dokumentationen 
finns även en beskrivning av programmet med vilket roboten kan styras trådlöst 
via en PC. 

För information angående användningen av Komborobot hänvisas till dokumentet \emph{Användarmanual}. 

\subsection{Bakgrund och syfte}
Konstruktionen av Komborobot har skett som en del i kursen Elektronikprojekt Y (TSEA28), vårterminen 2012. Grupp 17, som består av projektmedlemmarna på sid. 2, fick innan projektets start i uppdrag av beställaren Tomas Svensson, ISY, att konstruera en robot som uppfyller vissa specifierade krav (se kravspecifikation för produkten). Huvudmålet har varit att roboten autonomt skall kunna navigera genom en kombinerad linje/labyrintbana (Se banspecifikation, appendix A i kravspecifikationen).  Det har också varit önskvärt att få roboten att ta sig igenom banan på så kort tid som möjligt eftersom roboten vid projektets slut kommer att tävla mot en liknande robot, konstruerad av en annan projektgrupp, där snabbaste roboten vinner tävlingen.

