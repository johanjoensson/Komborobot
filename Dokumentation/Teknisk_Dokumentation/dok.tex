\section{Dokument}
Två dokument ska framställas och medfölja vid leveransen av roboten, en användarmanual samt teknisk dokumentation. All dokumentation skrivs enligt projektmodellen Lips.

\subsection{Användarmanual}
Användarmanualen ska fungera som en hjälp för den som använder den färdiga roboten. Den ska tydligt beskriva hur PC-gränssnittet fungerar och vilka kommandon som behövs för att styra roboten manuellt. Det ska även framgå hur man ställer om mellan autonomt och manuellt läge, samt under vilka förhållanden roboten kan förväntas kunna styra autonomt.

\subsection{Teknisk dokumentation}
Den tekniska dokumentationen ska till största del författas under tiden roboten konstrueras, och ska nogrannt beskriva varje delsystem. Dokumentationen är en del av slutleveransen, och det ska med hjälp av denna vara möjligt att bygga en kopia av roboten som konstrueras under projektet. Alla faser i konstruktionen som bidragit till den slutgiltiga roboten ska därmed dokumenteras och slutligen sammanställas till den tekniska dokumentationen.