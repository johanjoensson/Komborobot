\section{PC-mjukvara}

PC-mjukvaran har ett grafiskt gränssnitt som tillåter kommunikation mellan en
användare och roboten. Med PC-mjukvaran kan en användare: se kartan som roboten
ritar upp, styra roboten manuellt och läsa debug-information. PC-mjukvaran
kommunicerar med roboten via blåtand.

\subsection{Implementation}

PC-mjukvaran är skriven i programspråket Java. PC-mjukvaran använder Javas
standardbibliotek (J2SE) samt ett tredjepartsbibliotek med gränssnittet JSR 82
för blåtandskommunikation.

\subsection{Användande}

PC-mjukvaran består av ett program som visar ett fönster delat i två delar: en
karta och ruta för debug-information. Kartan uppdateras automatiskt när roboten
är i tävlingsläget. Programmet styrs med tangentbordet; \textit{F5} hämtar ny
debug-information som visas i debug-informationsfältet och om roboten är i
försatt i manuellt styrläge kommer \emph{piltangenterna} att styra roboten.

Programmet är lätt att använda; tangentbordsgenvägar och beskrivningar över vad
som visas på skärmen, syns i programfönstret.
