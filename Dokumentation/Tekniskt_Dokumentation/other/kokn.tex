\section{Kommunikations- och kartnavigeringsmodul}

Kartmodulen består av en Beagleboard xM och hanterar kartlogik, styr sensor-
och styrmodulen; samt kommunicerar med PC-mjukvaran.

\subsection{Hårdvara}
%Känns lite hurp durp, men det ska ju vara specat
Beagleboardens DC-anslutning kräver 5 V, 350 mA och är ansluten till
robotchassits inbyggda spänningsregulator.

För att implementera de två kontrollknapparna på roboten används knapparna på
en USB-mus användas. Mjukvaran fångar upp mushändelser. 

\subsection{Mjukvara}

\subsubsection{Operativsystem}

Vi ska använda standarddistributionen Ångström på Beagleboarden.

\subsubsection{Övrig mjukvara}

Under utvecklingsfasen används hjälpverktyg så som OpenSSH,
versionshanteringssystem, kompilatorer och textredigerare.

För att hantera USB- till serieomvandlaren och blåtandsstickan behövs
eventuellt drivrutiner.

Blåtandskommunikation hanteras med blåtandsstacken BlueZ. BlueZ möjliggör
för i princip vanlig sockelprogrammering för blåtandskommunikation i C.
% TODO ref http://people.csail.mit.edu/albert/bluez-intro/c404.html

För att hantera mushändelser används GPM.

\subsubsection{Egentuvecklad mjukvara}
Mjukvaran i kartmodulen är nästan uteslutande skriven i C. Mindre saker så som
start av robot och hantering av robotens program skrivs i shellskript.

Eftersom att kartmodulen styr roboten, hanterar reglerloop och kontinuerligt
kommunicerar med PC-mjukvaran, blir det svårt att implementera allt detta i ett
program. Mjukvaran är därför att uppdelad i olika program, vilket möjliggör
större utvecklingsparallellisering. Eftersom att roboten arbetar i två olika
lägen, fjärrstyrningsläget och det autonoma läget, så utvecklas dessa separat.

Lägesvalet hanteras av ett masterprogram. Masterprogrammet agerar även
mellanlager för I/O-hantering på roboten.

Högst prioritet i det autonoma läget har hanteringen av styr- och
sensormodulen. Blåtandshantering, d.v.s. att skicka över kartinformation till
datorn, är inte lika tidskritiskt. Mjukvaran för det autonoma läget är uppdelad
i två delar: dels ett huvudprogram som hanterar kartlogik, och dels ett program
för att hantera blåtandskommunikation med PC-mjukvaran. Kartlogikprogrammet
uppdaterar kontinuerligt en fill innehållandes kartdata som blåtandsprogrammet
läser av när PC-mjukvaran begär kartan.

Reglerloopen måste kontinuerligt efterfråga sensordata från sensormodulen. För
att reglerloopen ska fungera bäst så ska den ha ny sensordata för varje
uppdatering.

%\subsection{Blåtandshantering}
% Typ köra RFCOMM för att det är enkelt

\subsection{Kartalgoritmer}

Roboten måste dels kunna rita ut en kartan och hålla koll på sin
nuvarande position. Eftersom att kartan roboten ska ritas ut är har vissa
kraftigt begränsade egenskaper, så som väggdimensioner och endast 90-graders
vinklar, så kan implementationen av kartalgoritmen förenklas.

Kartalgoritmen ska skapa en förenklad matris som representerar 80x80 cm stora
block. För denna antas startpositionen vara i origo.

% Kolla FastSLAM, kanske lite överdriven för just vårt problem.
% http://tinyurl.com/5wesprc
