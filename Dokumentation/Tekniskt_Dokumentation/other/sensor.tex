\section{Sensormodulen}
Roboten har olika sensorer för att detektera väggar, sin rörelse och sin
rotation. 

\subsection{Hårdvara}
Sensormodulen baseras på en ATMega16, se Appendix \ref{sensor-sch} för
kopplingsschema.

\subsubsection{Analoga sensorer}
Vi har valt att använda tre stycken optiska avståndsmätare: en för 20-150 cm i
fronten och en på vardera sida för 10-80 cm. Deras utsignal är analog och denna
mäts med den inbyggda AD-omvandlaren i AVR:en.

Roboten har också hjulsensorer som mäter hjulens rotation. Dessa
sensorer är också inkopplade på AVR:ens AD-omvandlare. 

\subsubsection{Gyro}
Med ett gyro mäts robotens rotationshastighet. Utsignalen är en
tidspuls vars längd varierar. Denna signal mäts med hjälp av
sextonbitarsräknaren och avbrott i AVR:en.

\subsection{Mjukvara}
AVR:en kör ett minimalt realtidssytem skrivet i C. Efter initering
av AVR:en börjar en oändlig mätloop. Serieporten gör avbrott i mätloopen när
mätdata ska skickas varefter den är begärd.

Vid varje givet tillfälle har sensormodulen uppmätta värden från alla sensors,
och ersätta dem med så nya värden så ofta som möjligt.

\subsubsection{AD-omvandling}
Vi använder oss av omvandlarens fulla bredd, d.v.s. tio bitar, till
IR-avståndsmätare och hjulsensorer.

Alla sensorer är kopplade på olika ingångar till den analoga muxen, ADMUX, och
när ingång byts tar det upp till 125 $\mu$s innan värdet är stabiliserat.
AD-omvandlarens klocka går att köra i 200 kHz när den gör 10 bitars omvandling.
Vid första omvandligen tar det 25 AD-klocktick, vilket även det blir 125
$\mu$s. Det bör innebära en samplingsfrekvens på 4 kHz om alla omvandlingar är
lika långsamma som den första. Detta är nog för att mäta av alla fem analoga
sensorer flera hundra gånger per sekund.

\subsubsection{Tidsmätning}

Gyrots signal är som beskrivs i gyrots datablad
\footnote{\burl{https://docs.isy.liu.se/twiki/pub/VanHeden/DataSheets/gwspg03_gyro_prel.pdf}}.
För att mäta en puls ska INT0 vara satt att ge avbrott på fallande flank. När
det avbrottet kommer ska 16b-räknaren, TCNT1, startas och avbrottskriteriet ska
ändras till stigande flank. När det nya avbrottet kommer ska räknaren stoppas,
avläsas och nollställas. Därefter ska avbrottskriteriet återställas till
fallande flank.

\subsection{Utvärdering och kalibrering}
För att kunna få bra data på mätfel och individkalibrering av sensorerna så
skapar vi enkla repeterbara test. Vi testar rotationssensorer, avståndssensorer,
hjulsensorer.

Rotationssensorer testar vi lämpligen genom att rotera roboten en förutbestämd
mängd och mäta avvikelsen mellan olika försök.

Avsåndssensorer testas genom att mäta avstånd till en vägg och se hur stort fel
sensorn har och rotera roboten för att kontrollera hurvida vinkel påverkar 
storleken på felet.

Hjulsensorerna testas genom att roboten får åka olika streckor och differansen
mellan uppmätt avstånd och sensorns upfattning därom kontrolleras.