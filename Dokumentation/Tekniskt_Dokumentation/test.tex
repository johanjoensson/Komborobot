\section{Testning}
Tester av systemet ska göras både i större och mindre utförande för att undvika att tekniska problem och designmissar följer med under längre tid i projektet.

\subsection{Kontinuerliga tester}
Kontinuerliga tester ska ske regelbundet genom hela projektet. Dessa tester ska ske avslutningsvis efter varje aktivitet. Detta innebär att när en aktivitet i tidsplanen avslutats så ska tester utföras för att bekräfta funktion och huruvida de kraven som är satta uppfylls. Först när detta är gjort så ska resten av gruppen meddelas att aktiviteten är slutförd. Vill ansvarig för en aktivitet fortsätta och bygga vidare på en lösning som efter utförda tester visar sig inte uppfylla kraven för aktiviteten ifråga, så ska testresultaten rapporteras så att nästa gruppdeltagare som ska använda, eller utveckla, vet vilka fel som uppstod.

\subsection{Stora tester}
Tester i större skala ska göras i samband med att en modul är färdig. Testet ska då vara omfattande och alla kraven som påverkar modulen ska gås igenom grundligt. Speciellt viktigt är att testa de signaler och den kommunikation som sker med andra enheter. Skulle något av detta vara fel så måste det lösas samt rapporteras grundligt.

\subsection{Kalibrering}
När tester utförs för att kalibrera systemet så ska de värden som testats antecknas. Det ska även dokumenteras vilket utfall det fick och några kortare kommentarer om varför det är bra eller dåligt ska lämnas. Detta görs för att undvika att samma värden testas igen och för att förenkla för gruppen att se samband och hitta optimala värden.