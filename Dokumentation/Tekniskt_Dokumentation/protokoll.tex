\section{Protokoll}
\label{sec:Protokoll}
Här beskrivs hur kommunikationen mellan de olika delsystemen kommer att implementeras.

\subsection{Generell överföring mellan enheter}
Kommunikationen mellan de olika enheterna kommer att gå via en SPI-buss (se \ref{sec:SPI}).
All kommunikation består av en headerbyte följt av en eller flera databytes.
I headerbyten anges till vilken enhet den efterföljande informationen ska, samt hur den ska tolkas av den mottagande enheten.
De tre mest signifikanta bitarna i headern anger till vilken enhet informationen ska och de 5 minst signifikanta bitarna anger 
hur denna information ska behandlas. Uppdelningen kan ses i tabell \ref{tab:header}.
Kodningen av de tre mest signifikanta bitarna kan ses i tabell \ref{tab:headerkod}.

All kommunikation kommer att gå via kommunkationsenheten, så om sensorenheten vill 
skicka data till styrenheten skickar den en headerbyte med styrenhetens adress och en databyte till kommunikationsenheten som 
tolkar headern och skickar headern och databyten till styrenheten.
\begin{table}[h]
  \centering
  \begin{tabular}{| c | c |}
    \hline
    Målenhet [3 bitar] & Flaggor [5 bitar]\\
    \hline
  \end{tabular}
  \caption{Bituppdelning av header}
  \label{tab:header}
\end{table}

\begin{table}[h]
  \centering
  \begin{tabular}{l l}
    \textbf{3 MSB} & \textbf{Målenhet} \\
    000 & Kommunikationsenhet \\
    001 & Sensorenhet \\
    010 & Styrenhet \\
    100 & PC \\
    110 & PC och Styrenhet\\
  \end{tabular}
  \caption{Kodning av header}
  \label{tab:headerkod}
\end{table}


\subsection{PC-mjukvara $\longleftrightarrow$ Kommunikationsenhet}
PC-mjukvarn kommer generera en header-byte samt en data-byte som sedan skickas
till kommunikationsenheten via blåtand. Kommunikationsenheten kommer därefter
behandla de två byten precis som alla andra.

Då kommunikationsenheten sänder data till PC-mjukvaran kommer mjukvaran att
översätta header-byten samt data-byten så att man enkelt kan åskådliggöra
instruktionen som sändes.

\subsection{Sensorenhet $\longleftrightarrow$ Kommunikationsenhet}
Sensorenheten skickar data till kommunikationsenheten som ska vidare både till PCn och styrenheten. Detta anges med en header enligt tabell \ref{tab:headerkod}. Efterföljande data är antingen ett styrkommando, se tabell \ref{tab:styrkommando}, eller avstånd och riktning till centrumlinjen.

\subsection{Styrenhet $\longleftrightarrow$ Kommunikationsenhet}
Kommunikationen mellan styrenheten och kommunikationsenheten kommer att bete sig olika beroende på
om roboten är i fjärrstyrt eller autonomt läge.
\subsubsection{Fjärrstyrt läge}
I fjärrstyrt läge så kommer styrenheten att ta emot headers och styrkommandon. I detta läge så 
innehåller inte headern någon intressant information och kommer att ignoreras. Styrkommandots bituppdelning kan ses i 
tabell \ref{tab:styrbitar} och dess kodning i tabell \ref{tab:styrkommando}.
 
\begin{table}[h] 
  \centering
  \begin{tabular}{| c | c |}
    \hline
    Kommando [4 bitar] & Hastighet [4 bitar] \\ \hline
  \end{tabular}
  \caption{Bituppdelning av styrkommandon}
  \label{tab:styrbitar}
\end{table}

\begin{table}[h] 
  \centering
  \begin{tabular}{l l}
    \textbf{4 MSB} & \textbf{Kommando} \\
    0000 & Stopp \\
    0001 & Framåt \\
    0010 & Fram vänster \\
    0011 & Fram höger \\
    0100 & Rotera vänster \\
    0101 & Rotera höger \\
    0110 & Back \\
  \end{tabular}
  \caption{Kommandokodning}
  \label{tab:styrkommando}
\end{table}

\subsubsection{Autonomt läge}
I autonomt läge så avgör en bit i headern om efterföljande byte är sensordata som ska användas i regleringen eller
om det är ett specialkommando som ska utföras utan reglering.
Bituppdelning av kommandon ses i tabell \ref{tab:specialbitar} och kodningen i tabell \ref{tab:special}.

Eftersom en SPI-buss används så kommer även styrenheten att skicka data till kommunikationsenheten samtidigt som den tar emot data.
Detta kommer att användas till att skicka information om körningen, t ex ``Vanlig reglering'' eller ``Svänger vänster''.

\begin{table}[h] 
  \centering
  \begin{tabular}{| c | c |}
    \hline
    Specialkommando [3 bitar] & Argument [5 bitar] \\ \hline
  \end{tabular}
  \caption{Bituppdelning av specialkommandon}
  \label{tab:specialbitar}
\end{table}

\begin{table}[h]
  \centering
  \begin{tabular}{l l}
    \textbf{3 MSB} & \textbf{Specialkommando} \\
    000 & Återuppta vanlig reglering\\
    001 & Kör rakt fram \\
    010 & Sväng vänster 90\degree \\
    011 & Sväng höger 90\degree \\
  \end{tabular}
  \caption{Kodning av specialkommandon}
  \label{tab:special}
\end{table}
