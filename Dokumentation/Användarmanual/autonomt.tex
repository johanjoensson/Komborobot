% !TEX encoding = UTF-8 Unicode

%% --------------------------------------------------------------------------------------------------------------------------------------------
% AUTONOMT LÄGE - Beskrivning av användning vid autonom styrning 
%
%
% --mj 20120506
%% --------------------------------------------------------------------------------------------------------------------------------------------

\section{Autonomt läge}
I det autonoma läget så styr roboten sig själv. Roboten är designad för att kunna ta sig igenom en bana bestående av labyrinter samt linjer på marken. Banans utseende beskrivs i detalj i sektion \ref{sec:banan}.

\subsection{Start och stopp}
För att starta roboten i det autonoma läget så görs följande: (se figur \ref{fig:robot1} för nummerreferenser)

\begin{enumerate}
\item Slå på strömmen. (5)
\item Se till att roboten är ställd i autonomt läge. (1)
\item Tryck på reset för att sätta roboten i inaktivt läge. (3)
\item Roboten kan börja antingen på en linje eller i labyrinten. I linjeläge måste robotens linjesensorer stå över linjen, och roboten vara riktad i riktning med linjen. 
\item Körning startas med startknappen. (2)
\end{enumerate}

Roboten kommer att följa banan fram till dess att stoppkommando hittas. Man kan även stanna roboten manuellt med reset-kappen (3) eller genom att slå av strömmen (5). 

\subsection{Kalibrering}
Todo

\subsection{Displayen}
Displayen visar avståndet till väggarna på sidorna till höger, vänster och framför, givet i cm. Figur \ref{fig:display} visar displayens utseende. 

\nyBild{display1.jpg}{Displayens utseende.}{display}{1.0}

Displayen har två rader med tecken. Den övre raden visar framsensorns och de främre sidosensorernas värden, och den undre raden visar bakre sidosensorernas dito. Ett 'H' bredvid värdet indikerar högersensor, 'V' vänstersensor och 'F' sensorn riktad framåt. Avstånden visas med tre siffror, och ligger i intervallet [20 cm, 120 cm]. Detta innebär att avstånd som ligger utanför detta intervall kommer att ge ett felaktigt värde på displayen. 

\subsection{PC}
Todo

\subsection{Banan}
\label{sec:banan}
Banan som roboten är konstruerad för kan delas upp i två separata moment, en labyrint att navigera i och en linje på marken att följa. De två momenten 

\subsection{Övrigt}
